% \iffalse
%% Style file `jumoline'.
%% Copyright (C) 1999-2001  Hiroshi Nakashima
%%		       (Toyohashi Univ. of Tech.)
%% 
%% This program can be redistributed and/or modified under the terms
%% of the LaTeX Project Public License distributed from CTAN
%% archives in directory macros/latex/base/lppl.txt; either
%% version 1 of the License, or any later version.
%% 
% \fi
%
% \CheckSum{770}
%% \CharacterTable
%%  {Upper-case    \A\B\C\D\E\F\G\H\I\J\K\L\M\N\O\P\Q\R\S\T\U\V\W\X\Y\Z
%%   Lower-case    \a\b\c\d\e\f\g\h\i\j\k\l\m\n\o\p\q\r\s\t\u\v\w\x\y\z
%%   Digits        \0\1\2\3\4\5\6\7\8\9
%%   Exclamation   \!     Double quote  \"     Hash (number) \#
%%   Dollar        \$     Percent       \%     Ampersand     \&
%%   Acute accent  \'     Left paren    \(     Right paren   \)
%%   Asterisk      \*     Plus          \+     Comma         \,
%%   Minus         \-     Point         \.     Solidus       \/
%%   Colon         \:     Semicolon     \;     Less than     \<
%%   Equals        \=     Greater than  \>     Question mark \?
%%   Commercial at \@     Left bracket  \[     Backslash     \\
%%   Right bracket \]     Circumflex    \^     Underscore    \_
%%   Grave accent  \`     Left brace    \{     Vertical bar  \|
%%   Right brace   \}     Tilde         \~}
%%
% \iffalse
%%
%<*dtx>
\ProvidesFile{jumoline.dtx}
%</dtx>
%<package>\def\next{LaTeX2e}
%<package>\ifx\fmtname\next
%<package>\def\next{
%<package>\NeedsTeXFormat{LaTeX2e}[1994/12/01]
%<package>\ProvidesPackage{jumoline}}
%<package>\else\def\next[#1]{}\fi
%<package>\next
%<driver>\ProvidesFile{jumoline-man.tex}
% \fi
% \ProvidesFile{jumoline.dtx}
[2001/05/31 v1.2 ]
% \iffalse
%<*driver>
\documentclass{jltxdoc}
\usepackage{jumoline}
\DisableCrossrefs
\PageIndex
\CodelineNumbered
\RecordChanges
%</driver>
%<driver>\OnlyDescription
%<driver>\def\ONLYDESCRIPTION{}
%<*driver>
\advance\textwidth4em
\begin{document}
\DocInput{jumoline.dtx}
\end{document}
%</driver>
% \fi
% \changes{v0.1}{1995/06/06}
%	{関連スタイルjulem作成。}
% \changes{v0.2}{1996/10/05}
%	{関連スタイルoverstroke作成。}
% \changes{v1.0}{1999/04/14}
%	{機能/実装の根本的見直しと本資料追加。}
% \changes{v1.1-1}{2000/07/11}
%	{以下は本資料に関する改訂。}
% \changes{v1.2}{2001/05/31}
%	{本資料に関する改訂。}
%
% \errorcontextlines10000
% \ifx\xkanjiskip\undefined
% \def\!{\hskip\jintercharskip}
% \else
% \def\!{\hskip\xkanjiskip}
% \fi
% \def\|{\!\verb|}
% \def\+{\verb+}
% \DeleteShortVerb\|
% \GetFileInfo{jumoline.dtx}
% \setcounter{IndexColumns}{2}
% 
% \def\bottomfraction{.9}
% \def\abstractname{概要}
% \def\refname{参考文献}
% \def\indexname{索引}
% \def\figurename{図}
% \def\tablename{表}
% \def\contentsname{目次}
% 
% \def\cs#1{\hbox{\tt\char\escapechar#1}}
% \def\Item{\vskip-\itemsep\vskip-\parsep\vskip0pt\item}
%
% \makeatletter
% \def\eqnarray{\stepcounter{equation}\let\@currentlabel=\theequation
% \global\@eqnswtrue
% \global\@eqcnt\z@\tabskip\@centering\let\\=\@eqncr
% $$\halign to \displaywidth\bgroup\hskip\@centering
%   $\displaystyle\tabskip\z@{##}$\@eqnsel&\global\@eqcnt\@ne
%   \hfil$\;{##}\;$\hfil
%   &\global\@eqcnt\tw@ $\displaystyle\tabskip\z@{##}$\hfil
%    \tabskip\@centering&\llap{##}\tabskip\z@\cr}
%
% \def\Meta#1{\penalty\@highpenalty
%	\hbox{{\tt\char`\{}{\it\meta{#1}}{\tt\char`\}}}}
% \def\opt#1{{\tt[}{\it\meta{#1}}{\tt]}}
% \def\optn#1{{\tt[#1]}}
% \def\arg#1{{\tt\char`\{#1\char`\}}}
%
% \let\latex@addmarginpar\@addmarginpar
% \let\latex@latex@warning@no@line\@latex@warning@no@line
% \def\@addmarginpar{\let\@latex@warning@no@line\@gobble
%	\latex@addmarginpar
%	\let\@latex@warning@no@line\latex@latex@warning@no@line}
% \let\@font@warning\@gobble
%
% \long\def\m@cro@#1#2{\endgroup \topsep\MacroTopsep \trivlist
%    \edef\saved@macroname{\string#2}
%   \def\makelabel##1{\llap{##1\hskip\@totalleftmargin}}
%   \if@inlabel
%     \let\@tempa\@empty \count@\macro@cnt
%     \loop \ifnum\count@>\z@
%       \edef\@tempa{\@tempa\hbox{\strut}}\advance\count@\m@ne \repeat
%     \edef\makelabel##1{\llap{\vtop to\baselineskip
%                                {\@tempa\hbox{##1}\vss}
%			       \hskip\@totalleftmargin}}
%     \advance \macro@cnt \@ne
%   \else  \macro@cnt\@ne  \fi
%   \edef\@tempa{\noexpand\item[
%      #1
%        \noexpand\PrintMacroName
%      \else
%        \noexpand\PrintEnvName
%      \fi
%      {\string#2}]}
%   \@tempa
%   \global\advance\c@CodelineNo\@ne
%    #1
%       \SpecialMainIndex{#2}\nobreak
%       \DoNotIndex{#2}
%    \else
%       \SpecialMainEnvIndex{#2}\nobreak
%    \fi
%   \global\advance\c@CodelineNo\m@ne
%   \ignorespaces}
% 
% \def\DescribeOpt{\leavevmode\@bsphack\begingroup\MakePrivateLetters
%   \Describe@Opt}
% \def\Describe@Opt#1{\endgroup \Describe@pt{#1}}
% \def\Describe@pt#1#2{
%               \marginpar{\raggedleft\PrintDescribeMacro{#1}}
% 		\SpecialOptUsageIndex{#1}{#2}}
% \def\SpecialOptUsageIndex#1#2{
% 	\index{#1\actualchar{\protect\ttfamily#1} (#2)\encapchar usage}
%	\index{#2s:\levelchar{\protect\ttfamily#1}\encapchar usage}
%	\@esphack\ignorespaces}
% \def\SpecialOptMainIndex#1#2{
% 	\index{#1\actualchar{\protect\ttfamily#1} (#2)\encapchar main}
%	\index{#2s:\levelchar{\protect\ttfamily#1}\encapchar main}
%	\@esphack\ignorespaces}
% \def\SpecialOptIndex#1#2{
% 	\index{#1\actualchar{\protect\ttfamily#1} (#2)}
%	\index{#2s:\levelchar{\protect\ttfamily#1}}
%	\@esphack\ignorespaces}
% \def\SpecialOptIndexS#1#2{
% 	\index{#1\actualchar{\protect\ttfamily#1} (#2)}
%	\index{#2es:\levelchar{\protect\ttfamily#1}}
%	\@esphack\ignorespaces}
%
% \def\SpecialMainIndex#1{\@bsphack
%	\RemoveAtFromCS{#1}
% 	\protected@edef\@tempa{\noexpand\special@index{\RAFCtempa
%		\actualchar \string\verb
% 		\quotechar*\verbatimchar\string#1\verbatimchar
%		\encapchar main}}\@tempa
%	\@esphack}
% \def\SpecialIndex#1{\@bsphack
%	\RemoveAtFromCS{#1}
% 	\protected@edef\@tempa{\noexpand\special@index{\RAFCtempa
%		\actualchar \string\verb
% 		\quotechar*\verbatimchar\string#1\verbatimchar}}\@tempa
%	\@esphack}
% 
% \let\RAFCempty\@empty
% \let\RAFCgobble\@gobble
% \makeatother
% \def\RemoveAtFromCS#1{\edef\RAFCtempa{\expandafter\RAFCgobble\string#1}
% 	\edef\RAFCtempa{\expandafter\string\csname\RAFCtempa\endcsname}
%	\expandafter\iRemoveAtFromCS\RAFCtempa@\NIL}
% \def\iRemoveAtFromCS#1#2@#3\NIL{\def\RAFCtempb{#3}
% 	\ifx\RAFCtempb\RAFCempty \def\RAFCtempa{#2}\let\next\relax
%	\else \def\RAFCtempa{#2@}\def\next{\iiRemoveAtFromCS#3\NIL}
% 	\def\RAFCtempb{#2}
%	\ifx\RAFCtempb\RAFCempty
%	\else\ifx\RAFCtempb\CommonCSPrefix
%	\else\ifx\RAFCtempb\CommonIfPrefix
%	\else \def\RAFCtempa{}\def\next{\xRemoveAtFromCS#2@#3\NIL}\fi\fi\fi\fi
%	\next}
% \def\iiRemoveAtFromCS#1{\ifx#1@\edef\RAFCtempa{\RAFCtempa@}
%	\let\next\iiRemoveAtFromCS
%	\else \def\next{\xRemoveAtFromCS#1}\fi
% 	\next}
% \def\xRemoveAtFromCS#1@\NIL{\edef\RAFCtempa{#1\RAFCtempa}}
% \makeatletter
% 
% \begingroup
% \gdef\@sverbcat{\catcode`\@\active\catcode`\#\active}
% \catcode`\@\active \gdef@{\char`\@\penalty\exhyphenpenalty}
% \catcode`\#\active \gdef#{\penalty\exhyphenpenalty}
% \endgroup
% \let\latex@sverb\@sverb
% \def\@sverb{\@sverbcat\latex@sverb}
%
% \newcount\ChangesCounter \ChangesCounter10000
% \def\changes@#1#2#3{\global\advance\ChangesCounter\@ne
% 	\protected@edef\@tempa{\noexpand\glossary{#1\levelchar
%		x\number\ChangesCounter
%		\ifx\saved@macroname\@empty
%			\actualchar\generalname
%		\else
%			\expandafter\@gobble\saved@macroname
%			\actualchar \string\verb
%			\quotechar*\verbatimchar \saved@macroname
%			\verbatimchar
%		\fi
%		:\levelchar #3}}\@tempa\endgroup\@esphack}
% \def\theglossary{\onecolumn \glossary@prologue
%	\GlossaryParms \let\item\@idxitem \ignorespaces}
% \def\endtheglossary{\relax}
% 
% 
%
% \def\bracepair{`{\tt\char`\{\char`\}'}}
%
% \ifx\ONLYDESCRIPTION\undefined\else
% \let\SpecialUsageIndex\SpecialMainIndex
% \let\SpecialOptUsageIndex\SpecialOptMainIndex
% \fi
% 
% 
%
% \edef\CommonCSPrefix{\expandafter\RAFCgobble\string\UMO}
% \edef\CommonIfPrefix{\expandafter\RAFCgobble\string\ifUMO}
%
% \title{下線/中線/上線パッケージ\!\textsf{jumoline}\thanks{
%	本ファイルのバージョンは\fileversion, 改訂日付は\filedate である。}}
% \author{中島 浩\\(豊橋技術科学大学)}
% \date{\filedate}
% \maketitle
% 
% \begin{abstract}
% 本ファイルは,和文テキストに改行可能な下線(\+\Underline+)/中線
% (\+\Midline+)/上線(\+\Overline+)を施すためのマクロを提供する。
% \end{abstract}
%
% \tableofcontents
% \newpage
%
% 
% \section{はじめに}
% \label{sec:intro}
%
% \TeX や\LaTeX が提供する諸機能の中でしばしば問題となるものの一つとして,
% \|\underline|や\|\overline|により下線/上線を施したテキストの途中での改行
% ができないことがあげられる。したがって,たとえば文書の改訂箇所を明示するな
% どの目的で比較的長い文章の全てに下線を施すことは,\+\underline+では困難で
% ある。
%
% この問題の完全な解決策を見い出すことは極めて困難であるが,ある一定の制約,
% たとえば\!\textsf{jumoline}が提供するマクロのように下線対象を「普通のテキス
% ト」に限定するといった制約を課すれば,解決することは不可能ではない。実際,
% 磯崎氏によるスタイルファイル\!\textsf{juline}や,著者自身による改良版であ
% る\!\textsf{julem}\footnote{
%
% このスタイル名は,著者が\!\textsf{fj.comp.texhax}に投稿したマクロが流通す
% る過程で命名されたものであるらしく,その由来や意味は著者にも不明である。}
%
% が提供する\|\juline|は,同様の制約下での一定の解決策として流通している。し
% かし\|\juline|には,禁則処理や特殊記号の「追い込み」など和文特有の改行/
% スペーシング処理が,下線等を施していない「普通の」文章とは必ずしも一致しな
% いという問題点がある。
%
% \textsf{jumoline}はこのような問題点を可能な限り解決するものであり,下線等が
% 施された文章に対しても,通常の文章と(ほぼ)同じ改行/スペーシングを行なう
% ことを目的としている。
% 
% なお\textsf{jumoline}は,ASCII版ついては\LaTeX-2.09と\LaTeXe の双方に対応し
% ているが,NTT版では動作しない。
%
% 
% \section{使用法}
% \label{sec:usage}
%
% \subsection{基本的な使用法}
% \label{sec:usage-basic}
%
% パッケージの使用宣言は標準的なものであり,\LaTeXe であれば;
% \begin{quote}
% \+\usepackage{jumoline}+\SpecialIndex{\usepackage}
% \end{quote}
% とすればよく,日本語\LaTeX-2.09であれば;
% \begin{quote}
% \+\documentstyle[..,jumoline,...]+\Meta{style}\SpecialIndex{\documentstyle}
% \end{quote}
% とすればよい。
%
% 下線等を施す基本的なコマンドは,以下の3つである。
%
% \begin{itemize}
% \item \+\Underline+\Meta{text}
%	\DescribeMacro{\Underline}\par
% \meta{text}に下線を施す。\Underline{たとえばこの文章は,
% {\tt\string\Underline\char`\{た}とえば{\ldots}であ{る。\tt\char`\}}により
% 生成されたものである。}
%
% \item \+\Midline+\Meta{text}
%	\DescribeMacro{\Midline}\par
% \meta{text}に中線を施す。\Midline{たとえばこの文章は,
% {\tt\string\Midline\char`\{た}とえば{\ldots}であ{る。\tt\char`\}}により
% 生成されたものである。}
%
% \item \+\Overline+\Meta{text}
%	\DescribeMacro{\Overline}\par
% \meta{text}に上線を施す。\Overline{たとえばこの文章は,
% {\tt\string\Overline\char`\{た}とえば{\ldots}であ{る。\tt\char`\}}により
% 生成されたものである。}
% \end{itemize}
%
% いずれのコマンドについても,引数\!\meta{text}は以下のいずれかを要素とする
% 列である。
% \begin{itemize}
% \item
% いわゆる全角文字。たとえば;
%
% \begin{quote}
%\begin{verbatim}
%\Underline{この文章は全角文字だけを含んでいるので,特別な配慮をすることなく
%下線を施すことができる。}
%\end{verbatim}
% \end{quote}
%
% は,以下を生成する。
%
% \begin{quote}
% \Underline{この文章は全角文字だけを含んでいるので,特別な配慮をすることなく
% 下線を施すことができる。}
% \end{quote}
%
% \item
% いわゆる半角文字。ただしコマンド呼び出しや数式は除く。たとえば;
%
% \begin{quote}
%\begin{verbatim}
%\Underline{この文章は,macro invocationやmath staffを含まないので,やはり特
%別な配慮は不要である。}
%\end{verbatim}
% \end{quote}
%
% は,以下を生成する。
%
% \begin{quote}
% \Underline{この文章は,macro invocationやmath staffを含まないので,やはり特
% 別な配慮は不要である。}
% \end{quote}
%
% またもちろん;
% \begin{quote}
%\begin{verbatim}
%\Underline{English sentences like this also may be the argument of
%macros without any special treatments.}
%\end{verbatim}
% \end{quote}
%
% のように英文をそのまま引数とすることもでき,以下が生成される。
%
% \begin{quote}
% \Underline{English sentences like this also may be the argument of macros
% without any special treatments.}
% \end{quote}
%
% \item
% `\+{+'と`\+}+'で囲まれた任意の文字列。\LaTeX の\|\mbox|の引数となりうるも
% のであれば,どのようなものであってもよい。文字列は\|\mbox|に入れた形で処理
% されるので,文字列の途中で改行は行なわれない。たとえば;
%
% \begin{quote}
%\begin{verbatim}
%\Underline{この文章の{\textbf{ゴチック}}の部分や数式 {$y=x^2$} の部分は,
%普通ではない文字を含んでいるので波カッコで囲まれている。}
%\end{verbatim}
% \end{quote}
%
% は,以下を生成する。
% \begin{quote}
% \Underline{この文章の{\textbf{ゴチック}}の部分や数式 {$y=x^2$} の部分は,
% 普通ではない文字を含んでいるので波カッコで囲まれている。}
% \end{quote}
%
% \item
% スペース/改行制御のための\|\UMOspace|, \+\UMOnewline+, および\|\\|。詳細
% は後述。
% \end{itemize}
% 
% 
%
% \subsection{描線の制御}
% \label{sec:usage-line}
%
% \+\Underline+などで描かれる線のデフォルト位置は,ベースラインを基準として
% 以下のように定まっている。ただし$b$は
% \|\baselineskip|\SpecialIndex{\baselineskip}の値,$a$は文字「あ」の高さで
% あり,それぞれの値は\|\Underline|などが実行される時点でのものである。また
% 正値は上方,負値は下方をそれぞれ意味する。
% \SpecialIndex{\strutbox}
%
% \begin{eqnarray*}
% \+\Underline+&:&	-0.3b\ \hbox{(\+\strutbox+の深さ)}\\
% \+\Midline+&:&	+0.5a\\
% \+\Overline+&:&	+0.7b\ \hbox{(\+\strutbox+の高さ)}\\
% \end{eqnarray*}
%
% これらのデフォルト以外の位置に描線したいときには,以下に示す長さコマンド
% (\TeX 流にいえばdimen register)を\|\setlength|\SpecialIndex{\setlength}
% 等によってベースラインからの距離(正値)に設定すればよい。
%
% \begin{itemize}
% \item
% \makebox[12em]{\cs{UnderlineDepth} \dotfill}
% \+\Underline+の深さ
% \DescribeMacro{\UnderlineDepth}
%
% \item
% \makebox[12em]{\cs{MidlineHeight} \dotfill}
% \+\Midline+の高さ
% \DescribeMacro{\MidlineHeight}
%
% \item
% \makebox[12em]{\cs{OverlineHeight} \dotfill}
% \+\Overline+の高さ
% \DescribeMacro{\OverlineHeight}
% \end{itemize}
%
% またデフォルトに戻すときには,任意の負値を設定すればよい。
%
% \changes{v1.2}{2001/05/31}
%	{\cs{UMOline}を\cs{Jumoline}としていた誤りを訂正。}
%
% \leavevmode\DescribeMacro{\UMOline}
% 描線位置制御の別の方法として;
%
% \begin{quote}
% \+\UMOline+\Meta{height}\Meta{text}
% \end{quote}
%
% も用意されている。このコマンドは\!\meta{text}を対象に,ベースラインから
% \!\meta{height}の高さ(負値は深さ)に描線する。
%
% \leavevmode\DescribeMacro{\UMOlineThickness}
% 描線の太さは長さコマンド\|\UMOlineThickness|に設定された値に従い,デフォル
% ト値は0.4\,ptである。
%
% 
%
% \subsection{スペース/改行の制御}
% \label{sec:usage-space}
%
% \+\Underline+などの引数\!\meta{text}がコマンド呼び出しを含む場合,
% \ref{sec:usage-basic}で述べたようにコマンドを`\+{+'と`\+}+'で囲まなければ
% ならない。したがって\|{\hspace|\Meta{len}\+}+\SpecialIndex{\hspace}によって挿
% 入した空白は,その直前で改行が行なわれても削除されない。また,
% \+\(no)linebreak+\SpecialIndex{\linebreak}\SpecialIndex{\nolinebreak}や
% \|\newline|\SpecialIndex{\newline}などによる改行制御は無効である。
%
% そこで\!\meta{text}の要素として,以下のスペース/改行制御コマンドが用意さ
% れている。
%
% \begin{itemize}
% \item
% \+\UMOspace+\Meta{len}\opt{num}
%	\DescribeMacro{\UMOspace}\par
% 直前で改行されると消去されるような,幅\!\meta{len}の空白を挿入する。
% オプション引数\!\meta{num}は,空白の直前での改行の困難さ/容易さを定める。
% \meta{num}が非負値($0\sim4$)の場合,
% \|\nolinebreak|\SpecialIndex{\nolinebreak}の引数と同じ意味で改行の困難さが
% 指定される。負値(${-0}\sim{-4}$)は改行の容易さを示し,その絶対値が
% \+\linebreak+\SpecialIndex{\linebreak}の引数の値に対応する。
%
% なお\!\opt{num}の直前に`\+*+'がある場合,\meta{num}は\TeX の
% \|\penalty|\SpecialIndex{\penalty}の値と解釈される。
%
% \item
% \+\UMOspace*+\Meta{len}
%	\DescribeMacro{\UMOspace*}\par
% 直前で改行されても消去されない,幅\!\meta{len}の空白を挿入する。
%
% \item
% \+\UMOnewline+, \+\\+
%	\DescribeMacro{\UMOnewline}
%	\DescribeMacro{\\}\par
% 改行する。改行によって生じた余白には下線などは施されない。
%
% \item
% \+\UMOnewline*+, \+\\*+
%	\DescribeMacro{\UMOnewline*}
%	\DescribeMacro{\\*}\par
%  改行する。改行によって生じた余白に下線などを施す。
% \end{itemize}
%
% 
%
% \subsection{注意事項}
% \label{sec:usage-note}
%
% \begin{enumerate}
% \item
% 欧文に対するハイフネーションは一切行なわれない。
%
% \item
% 一般の\LaTeX コマンドと同様,\!\meta{text}の中に
% \+\verb+\SpecialIndex{\verb}が含まれると(`\+{+'と`\+}+'に囲まれていても)
% エラーになることがある。
%
% \item
% 多くの\LaTeX コマンドと同様,\!\meta{text}の中にパラグラフの切れ目が含まれ
% ているとエラーになる。
%
% \item
% 和文/欧文スペーシングが,下線等を施さない場合と完全には一致しないことがあ
% り,その結果パラグラフの組み立てが変化することもある。
%
% \item
% \+\Underline+などが呼び出された時点,あるいは\+\Underline+が含まれるパラグ
% ラフ/ボックスが終了した時点で,\+\noautospacing+によって文字間スペーシン
% グが抑止されていても,\meta{text}は\|\autospacing|でスペーシングされる。た
% だし\bracepair に囲まれた部分については\|\noautospacing|となる。
% \SpecialIndex{\noautospacing}\SpecialIndex{\autospacing}
% 
% \item
% 下線等を施す処理は,通常の和文テキスト処理の100倍程度の時間を要し,またdvi
% ファイルの大きさも10倍程度に増加する。
% \end{enumerate}
%
% 
%
% \IndexPrologue{\newpage\section*{索引}
% 下線のページは項目の仕様記述箇所を示す。}
% \StopEventually{\PrintIndex}
%
% 
%
% \newpage
% \section{実装}
% \label{sec:imp}
%
% \subsection{問題点と解決策}
% \label{sec:imp-solution}
%
% 下線等の対象\!\meta{text}の要素を「普通の文字」と\bracepair に囲まれたもの
% (以下blocked textと呼ぶ)に限定することによって,要素を一つずつ切り出して
% 下線等を施すことが可能になるため,処理の根本的な困難さは解決される。しかし
% 次に問題となるのは,パラグラフ組み立て時に伸縮するスペース(glue)の取り扱い
% であり,特に和文では容易ではない。
% 
% すなわち問題点は;
% \begin{enumerate}
% \item
% 挿入されるglueの幅と伸縮項(すなわちnatural width\slash stretch\slash
% shrink)をどのようにして得るか。
% 
% \item
% そのglueと同じ幅と伸縮項を持つような下線等をどのようして描くか。
% \end{enumerate}
% である。
%
% この二つの問題点の内,後者は\|\leaders|を用いることにより比較的容易に解決
% できる。すなわち(あまり知られていないが);
% \SpecialIndex{\leaders}
%
% \begin{quote}
% \+\leaders\hrule height+\meta{$h$}\+depth+\meta{$d$}\+\hskip+\meta{$s$}
% \end{quote}
%
% において,$h$と$d$は必ずしも非負値である必要はなく,$h$はベースラインを基
% 準とした横線の上端までの距離(正値が上方)を,また$d$は下端までの距離(正
% 値が下方)を,それぞれ定めるものとして扱われる\footnote{
%
% この事実を認識していなかったため,\textsf{julem}では極めて短い線分からなる
% \!\cs{hbox}を要素とする\cs{leaders}によって下線を描くという,非効率な方法
% を採っていた。}。
% 
% したがって,上端がベースラインから下方へ$u$,太さが$w$であるような下線を描
% く場合;
%
% \begin{quote}
% \+\leaders\hrule height+\meta{$-u$}\+depth+\meta{$u{+}w$}\+\hskip+\meta{s}
% \end{quote}
%
% とすれば,伸縮や改行効果が\+\hskip+\meta{$s$}と全く同じ下線が得られる。ま
% た中線や上線についても同様である。
%
% \begin{table}[b]
% \caption{欧文スペースのglue}
% \label{tab:space}
% $$\arraycolsep0pt
% \begin{array}{r@{\;\land\;}l@{\quad}c@{\quad}c@{\quad}c}
% \multicolumn2c{\hbox{condition}}&n(\sigma)&p(\sigma)&m(\sigma)\\\hline
% f<2000&n(s)=0&	d_2&	d_3\cdot(f/1000)&	d_4\cdot(1000/f)\\
% f<2000&n(s)\neq0&	n(s)&	p(s)\cdot(f/1000)&	m(s)\cdot(1000/f)\\
% f\geq2000&n(x)=0&	d_2+d_7&d_3\cdot(f/1000)&	d_4\cdot(1000/f)\\
% f\geq2000&n(x)\neq0&	n(x)&	p(x)&			m(x)
% \end{array}
% $$
% \end{table}
%
% 一方前者については,欧文の場合は\TeX のスペーシング機構をシミュレートする
% ことにより,幅と伸縮項を算出することができる。すなわち\!\meta{text}に出現
% する(\bracepair に囲まれていない)各々のスペースについて,その直前の要素
% で定まる\|\spacefactor|の値を$f$, 選択されているフォントの
% \+\fontdimen+\meta{$i$}の値を$d_i$, \+\spaceskip+の値を$s$, \+\xspaceskip+
% の値を$x$, またglue $g$の幅,伸び,縮みの項をそれぞれ$n(g)$, $p(g)$,
% $m(g)$とすると,スペースにより挿入されるglue $\sigma$は表\ref{tab:space}に
% 示すように算出される。
% \SpecialIndex{\spacefactor}\SpecialIndex{\spaceskip}
% \SpecialIndex{\xspaceskip}
%
% 一方和文の場合,ASCII版は特に複雑であり,節を改めて議論する。
%
% 
%
% \begin{figure}[p]
% \makeatletter
% \parindent\z@ \@tempcntb\z@
% \newdimen\gna \newdimen\gsa \newdimen\gnb \newdimen\gsb
% \gna\z@ \gsa\z@ \gnb\z@ \gsb\z@
% \newcount\nw \@tempdima1zw \nw\@tempdima \divide\nw1000
% \def\Tenmin#1\\{\ifcase\@tempcntb
% 	\begingroup\advance\@tempcntb\@ne \or
% 	\advance\@tempcntb\@ne \def\C{#1}\or
% 	\putglue \endgroup\fi\ignorespaces}
% \def\glue#1 minus #2\\{
% 	\ifnum\@tempcntb=\@ne \gna#1pt\relax \gsa#2pt\relax
% 	\else		\gnb#1pt\relax \gsb#2pt\relax\fi\ignorespaces}
% \def\putglue{
% 	\settowidth\unitlength{\inhibitglue\C\inhibitglue}
% 	\ifdim\unitlength=1zw\else
% 	\leavevmode\hbox to.33\hsize{\fbox{
% 			\hskip\gna\inhibitglue\C\inhibitglue\hskip\gnb}\space
% 		\putg\gna$-$\putg\gsa\,/\,\putg\unitlength\,/\,
% 		\putg\gnb$-$\putg\gsb\hfil}\allowbreak\hfill\fi}
% \def\putg#1{\@tempcnta#1\divide\@tempcnta\nw
% 	\phantom{000}\llap{\number\@tempcnta}}
% 
% \Tenmin あ\\
% \Tenmin 、\\
% \glue 4.58203 minus 2.291\\
% \Tenmin あ\\
% \Tenmin あ\\
% \Tenmin 。\\
% \glue 4.58203 minus 2.291\\
% \Tenmin あ\\
% \Tenmin あ\\
% \Tenmin ,\\
% \glue 6.08551 minus 2.291\\
% \Tenmin あ\\
% \Tenmin あ\\
% \Tenmin .\\
% \glue 6.08551 minus 2.291\\
% \Tenmin あ\\
% \Tenmin あ\\
% \glue 3.04276 minus 3.04276\\
% \Tenmin ・\\
% \glue 3.04276 minus 3.04276\\
% \Tenmin あ\\
% \Tenmin あ\\
% \glue 3.04276 minus 3.04276\\
% \Tenmin :\\
% \glue 3.04276 minus 3.04276\\
% \Tenmin あ\\
% \Tenmin あ\\
% \glue 3.04276 minus 3.04276\\
% \Tenmin ;\\
% \glue 3.04276 minus 3.04276\\
% \Tenmin あ\\
% \Tenmin あ\\
% \glue 2.291 minus 2.291\\
% \Tenmin ?\\
% \glue 2.291 minus 2.291\\
% \Tenmin あ\\
% \Tenmin あ\\
% \glue 3.04276 minus 3.04276\\
% \Tenmin !\\
% \glue 3.04276 minus 3.04276\\
% \Tenmin あ\\
% \Tenmin あ\\
% \Tenmin ゛\\
% \glue 6.08551 minus 2.291\\
% \Tenmin あ\\
% \Tenmin あ\\
% \Tenmin ゜\\
% \glue 6.08551 minus 2.291\\
% \Tenmin あ\\
% \Tenmin あ\\
% \glue 3.04276 minus 3.04276\\
% \Tenmin ´\\
% \glue 3.04276 minus 3.04276\\
% \Tenmin あ\\
% \Tenmin あ\\
% \glue 3.04276 minus 3.04276\\
% \Tenmin `\\
% \glue 3.04276 minus 3.04276\\
% \Tenmin あ\\
% \Tenmin あ\\
% \glue 2.291 minus 2.291\\
% \Tenmin ¨\\
% \glue 2.291 minus 2.291\\
% \Tenmin あ\\
% \Tenmin あ\\
% \glue 2.291 minus 2.291\\
% \Tenmin ^\\
% \glue 2.291 minus 2.291\\
% \Tenmin あ\\
% \Tenmin あ\\
% \Tenmin ‾\\
% \Tenmin あ\\
% \Tenmin あ\\
% \Tenmin _\\
% \Tenmin あ\\
% \Tenmin あ\\
% \glue 2.291 minus 2.291\\
% \Tenmin ヽ\\
% \glue 2.291 minus 2.291\\
% \Tenmin あ\\
% \Tenmin あ\\
% \Tenmin ヾ\\
% \Tenmin あ\\
% \Tenmin あ\\
% \glue 2.291 minus 2.291\\
% \Tenmin ゝ\\
% \glue 2.291 minus 2.291\\
% \Tenmin あ\\
% \Tenmin あ\\
% \glue 1.07391 minus 1.07391\\
% \Tenmin ゞ\\
% \glue 1.07391 minus 1.07391\\
% \Tenmin あ\\
% \Tenmin あ\\
% \glue 1.07391 minus 1.07391\\
% \Tenmin 〃\\
% \glue 1.07391 minus 1.07391\\
% \Tenmin あ\\
% \Tenmin あ\\
% \Tenmin 仝\\
% \Tenmin あ\\
% \Tenmin あ\\
% \glue 1.07391 minus 1.07391\\
% \Tenmin 々\\
% \glue 1.07391 minus 1.07391\\
% \Tenmin あ\\
% \Tenmin あ\\
% \Tenmin 〆\\
% \Tenmin あ\\
% \Tenmin あ\\
% \Tenmin 〇\\
% \Tenmin あ\\
% \Tenmin あ\\
% \Tenmin ー\\
% \Tenmin あ\\
% \Tenmin あ\\
% \Tenmin —\\
% \Tenmin あ\\
% \Tenmin あ\\
% \glue 3.04276 minus 3.04276\\
% \Tenmin ‐\\
% \glue 3.04276 minus 3.04276\\
% \Tenmin あ\\
% \Tenmin あ\\
% \Tenmin /\\
% \Tenmin あ\\
% \Tenmin あ\\
% \Tenmin \\\
% \Tenmin あ\\
% \Tenmin あ\\
% \Tenmin 〜\\
% \Tenmin あ\\
% \Tenmin あ\\
% \glue 3.04276 minus 3.04276\\
% \Tenmin ‖\\
% \glue 3.04276 minus 3.04276\\
% \Tenmin あ\\
% \Tenmin あ\\
% \glue 3.04276 minus 3.04276\\
% \Tenmin |\\
% \glue 3.04276 minus 3.04276\\
% \Tenmin あ\\
% \Tenmin あ\\
% \Tenmin …\\
% \Tenmin あ\\
% \Tenmin あ\\
% \Tenmin ‥\\
% \Tenmin あ\\
% \Tenmin あ\\
% \glue 3.04276 minus 3.04276\\
% \Tenmin ‘\\
% \glue 3.04276 minus 3.04276\\
% \Tenmin あ\\
% \Tenmin あ\\
% \glue 3.04276 minus 3.04276\\
% \Tenmin ’\\
% \glue 3.04276 minus 3.04276\\
% \Tenmin あ\\
% \Tenmin あ\\
% \glue 2.291 minus 2.291\\
% \Tenmin “\\
% \glue 2.291 minus 2.291\\
% \Tenmin あ\\
% \Tenmin あ\\
% \glue 2.291 minus 2.291\\
% \Tenmin ”\\
% \glue 2.291 minus 2.291\\
% \Tenmin あ\\
% \Tenmin あ\\
% \glue 4.58203 minus 2.291\\
% \Tenmin (\\
% \Tenmin あ\\
% \Tenmin あ\\
% \Tenmin )\\
% \glue 4.58203 minus 2.291\\
% \Tenmin あ\\
% \Tenmin あ\\
% \glue 4.58203 minus 2.291\\
% \Tenmin 〔\\
% \Tenmin あ\\
% \Tenmin あ\\
% \Tenmin 〕\\
% \glue 4.58203 minus 2.291\\
% \Tenmin あ\\
% \Tenmin あ\\
% \glue 4.58203 minus 2.291\\
% \Tenmin [\\
% \Tenmin あ\\
% \Tenmin あ\\
% \Tenmin ]\\
% \glue 4.58203 minus 2.291\\
% \Tenmin あ\\
% \Tenmin あ\\
% \glue 4.58203 minus 2.291\\
% \Tenmin {\\
% \Tenmin あ\\
% \Tenmin あ\\
% \Tenmin }\\
% \glue 4.58203 minus 2.291\\
% \Tenmin あ\\
% \Tenmin あ\\
% \glue 4.58203 minus 2.291\\
% \Tenmin 〈\\
% \Tenmin あ\\
% \Tenmin あ\\
% \Tenmin 〉\\
% \glue 4.58203 minus 2.291\\
% \Tenmin あ\\
% \Tenmin あ\\
% \glue 4.58203 minus 2.291\\
% \Tenmin 《\\
% \Tenmin あ\\
% \Tenmin あ\\
% \Tenmin 》\\
% \glue 4.58203 minus 2.291\\
% \Tenmin あ\\
% \Tenmin あ\\
% \glue 4.58203 minus 2.291\\
% \Tenmin 「\\
% \Tenmin あ\\
% \Tenmin あ\\
% \Tenmin 」\\
% \glue 4.58203 minus 2.291\\
% \Tenmin あ\\
% \Tenmin あ\\
% \glue 4.58203 minus 2.291\\
% \Tenmin 『\\
% \Tenmin あ\\
% \Tenmin あ\\
% \Tenmin 』\\
% \glue 4.58203 minus 2.291\\
% \Tenmin あ\\
% \Tenmin あ\\
% \glue 4.58203 minus 2.291\\
% \Tenmin 【\\
% \Tenmin あ\\
% \Tenmin あ\\
% \Tenmin 】\\
% \glue 4.58203 minus 2.291\\
% \Tenmin あ\\
% \Tenmin あ\\
% \Tenmin +\\
% \Tenmin あ\\
% \Tenmin あ\\
% \Tenmin −\\
% \Tenmin あ\\
% \Tenmin あ\\
% \Tenmin ±\\
% \Tenmin あ\\
% \Tenmin あ\\
% \Tenmin ×\\
% \Tenmin あ\\
% \Tenmin あ\\
% \Tenmin ÷\\
% \Tenmin あ\\
% \Tenmin あ\\
% \Tenmin =\\
% \Tenmin あ\\
% \Tenmin あ\\
% \Tenmin ≠\\
% \Tenmin あ\\
% \Tenmin あ\\
% \Tenmin <\\
% \Tenmin あ\\
% \Tenmin あ\\
% \Tenmin >\\
% \Tenmin あ\\
% \Tenmin あ\\
% \Tenmin ≦\\
% \Tenmin あ\\
% \Tenmin あ\\
% \Tenmin ≧\\
% \Tenmin あ\\
% \Tenmin あ\\
% \Tenmin ∞\\
% \Tenmin あ\\
% \Tenmin あ\\
% \glue 1.07391 minus 1.07391\\
% \Tenmin ∴\\
% \glue 1.07391 minus 1.07391\\
% \Tenmin あ\\
% \Tenmin あ\\
% \glue 1.07391 minus 1.07391\\
% \Tenmin ♂\\
% \glue 1.07391 minus 1.07391\\
% \Tenmin あ\\
% \Tenmin あ\\
% \glue 1.07391 minus 1.07391\\
% \Tenmin ♀\\
% \glue 1.07391 minus 1.07391\\
% \Tenmin あ\\
% \Tenmin あ\\
% \glue 2.291 minus 2.291\\
% \Tenmin °\\
% \glue 2.291 minus 2.291\\
% \Tenmin あ\\
% \Tenmin あ\\
% \glue 2.291 minus 2.291\\
% \Tenmin ′\\
% \glue 2.291 minus 2.291\\
% \Tenmin あ\\
% \Tenmin あ\\
% \glue 2.291 minus 2.291\\
% \Tenmin ″\\
% \glue 2.291 minus 2.291\\
% \Tenmin あ\\
% \Tenmin あ\\
% \Tenmin ℃\\
% \Tenmin あ\\
% \Tenmin あ\\
% \Tenmin ¥\\
% \Tenmin あ\\
% \Tenmin あ\\
% \glue 1.07391 minus 1.07391\\
% \Tenmin $\\
% \glue 1.07391 minus 1.07391\\
% \Tenmin あ\\
% \Tenmin あ\\
% \glue 1.07391 minus 1.07391\\
% \Tenmin ¢\\
% \glue 1.07391 minus 1.07391\\
% \Tenmin あ\\
% \Tenmin あ\\
% \glue 1.07391 minus 1.07391\\
% \Tenmin £\\
% \glue 1.07391 minus 1.07391\\
% \Tenmin あ\\
% \Tenmin あ\\
% \Tenmin %\\
% \Tenmin あ\\
% \Tenmin あ\\
% \Tenmin #\\
% \Tenmin あ\\
% \Tenmin あ\\
% \Tenmin &\\
% \Tenmin あ\\
% \Tenmin あ\\
% \Tenmin *\\
% \Tenmin あ\\
% \Tenmin あ\\
% \Tenmin @\\
% \Tenmin あ\\
% \Tenmin あ\\
% \glue 2.291 minus 2.291\\
% \Tenmin §\\
% \glue 2.291 minus 2.291\\
% \Tenmin あ\\
% \Tenmin あ\\
% \Tenmin ☆\\
% \Tenmin あ\\
% \Tenmin あ\\
% \Tenmin ★\\
% \Tenmin あ\\
% \Tenmin あ\\
% \Tenmin ○\\
% \Tenmin あ\\
% \Tenmin あ\\
% \Tenmin ●\\
% \Tenmin あ\\
% \Tenmin あ\\
% \Tenmin ◎\\
% \Tenmin あ\\
% \Tenmin あ\\
% \Tenmin ◇\\
% \Tenmin あ\\
% \Tenmin あ\\
% \Tenmin ◆\\
% \Tenmin あ\\
% \Tenmin あ\\
% \Tenmin □\\
% \Tenmin あ\\
% \Tenmin あ\\
% \Tenmin ■\\
% \Tenmin あ\\
% \Tenmin あ\\
% \Tenmin △\\
% \Tenmin あ\\
% \Tenmin あ\\
% \Tenmin ▲\\
% \Tenmin あ\\
% \Tenmin あ\\
% \Tenmin ▽\\
% \Tenmin あ\\
% \Tenmin あ\\
% \Tenmin ▼\\
% \Tenmin あ\\
% \Tenmin あ\\
% \Tenmin ※\\
% \Tenmin あ\\
% \Tenmin あ\\
% \Tenmin 〒\\
% \Tenmin あ\\
% \Tenmin あ\\
% \Tenmin →\\
% \Tenmin あ\\
% \Tenmin あ\\
% \Tenmin ←\\
% \Tenmin あ\\
% \Tenmin あ\\
% \Tenmin ↑\\
% \Tenmin あ\\
% \Tenmin あ\\
% \Tenmin ↓\\
% \Tenmin あ\\
% \Tenmin あ\\
% \Tenmin 〓\\
% \Tenmin あ\\
% \Tenmin あ\\
% \Tenmin ∈\\
% \Tenmin あ\\
% \Tenmin あ\\
% \Tenmin ∋\\
% \Tenmin あ\\
% \Tenmin あ\\
% \Tenmin ⊆\\
% \Tenmin あ\\
% \Tenmin あ\\
% \Tenmin ⊇\\
% \Tenmin あ\\
% \Tenmin あ\\
% \Tenmin ⊂\\
% \Tenmin あ\\
% \Tenmin あ\\
% \Tenmin ⊃\\
% \Tenmin あ\\
% \Tenmin あ\\
% \Tenmin ∪\\
% \Tenmin あ\\
% \Tenmin あ\\
% \Tenmin ∩\\
% \Tenmin あ\\
% \Tenmin あ\\
% \Tenmin ∧\\
% \Tenmin あ\\
% \Tenmin あ\\
% \Tenmin ∨\\
% \Tenmin あ\\
% \Tenmin あ\\
% \Tenmin ¬\\
% \Tenmin あ\\
% \Tenmin あ\\
% \Tenmin ⇒\\
% \Tenmin あ\\
% \Tenmin あ\\
% \Tenmin ⇔\\
% \Tenmin あ\\
% \Tenmin あ\\
% \Tenmin ∀\\
% \Tenmin あ\\
% \Tenmin あ\\
% \Tenmin ∃\\
% \Tenmin あ\\
% \Tenmin あ\\
% \Tenmin ∠\\
% \Tenmin あ\\
% \Tenmin あ\\
% \Tenmin ⊥\\
% \Tenmin あ\\
% \Tenmin あ\\
% \Tenmin ⌒\\
% \Tenmin あ\\
% \Tenmin あ\\
% \Tenmin ∂\\
% \Tenmin あ\\
% \Tenmin あ\\
% \Tenmin ∇\\
% \Tenmin あ\\
% \Tenmin あ\\
% \Tenmin ≡\\
% \Tenmin あ\\
% \Tenmin あ\\
% \Tenmin ≒\\
% \Tenmin あ\\
% \Tenmin あ\\
% \Tenmin ≪\\
% \Tenmin あ\\
% \Tenmin あ\\
% \Tenmin ≫\\
% \Tenmin あ\\
% \Tenmin あ\\
% \Tenmin √\\
% \Tenmin あ\\
% \Tenmin あ\\
% \Tenmin ∽\\
% \Tenmin あ\\
% \Tenmin あ\\
% \Tenmin ∝\\
% \Tenmin あ\\
% \Tenmin あ\\
% \Tenmin ∵\\
% \Tenmin あ\\
% \Tenmin あ\\
% \Tenmin ∫\\
% \Tenmin あ\\
% \Tenmin あ\\
% \Tenmin ∬\\
% \Tenmin あ\\
% \Tenmin あ\\
% \Tenmin Å\\
% \Tenmin あ\\
% \Tenmin あ\\
% \Tenmin ‰\\
% \Tenmin あ\\
% \Tenmin あ\\
% \Tenmin ♯\\
% \Tenmin あ\\
% \Tenmin あ\\
% \Tenmin ♭\\
% \Tenmin あ\\
% \Tenmin あ\\
% \Tenmin ♪\\
% \Tenmin あ\\
% \Tenmin あ\\
% \Tenmin †\\
% \Tenmin あ\\
% \Tenmin あ\\
% \Tenmin ‡\\
% \Tenmin あ\\
% \Tenmin あ\\
% \Tenmin ¶\\
% \Tenmin あ\\
% \Tenmin あ\\
% \Tenmin ◯\\
% \Tenmin あ\\
% \Tenmin あ\\
% \Tenmin 0\\
% \Tenmin あ\\
% \Tenmin あ\\
% \Tenmin 1\\
% \Tenmin あ\\
% \Tenmin あ\\
% \Tenmin 2\\
% \Tenmin あ\\
% \Tenmin あ\\
% \Tenmin 3\\
% \Tenmin あ\\
% \Tenmin あ\\
% \Tenmin 4\\
% \Tenmin あ\\
% \Tenmin あ\\
% \Tenmin 5\\
% \Tenmin あ\\
% \Tenmin あ\\
% \Tenmin 6\\
% \Tenmin あ\\
% \Tenmin あ\\
% \Tenmin 7\\
% \Tenmin あ\\
% \Tenmin あ\\
% \Tenmin 8\\
% \Tenmin あ\\
% \Tenmin あ\\
% \Tenmin 9\\
% \Tenmin あ\\
% \Tenmin あ\\
% \Tenmin A\\
% \Tenmin あ\\
% \Tenmin あ\\
% \Tenmin B\\
% \Tenmin あ\\
% \Tenmin あ\\
% \Tenmin C\\
% \Tenmin あ\\
% \Tenmin あ\\
% \Tenmin D\\
% \Tenmin あ\\
% \Tenmin あ\\
% \Tenmin E\\
% \Tenmin あ\\
% \Tenmin あ\\
% \Tenmin F\\
% \Tenmin あ\\
% \Tenmin あ\\
% \Tenmin G\\
% \Tenmin あ\\
% \Tenmin あ\\
% \Tenmin H\\
% \Tenmin あ\\
% \Tenmin あ\\
% \Tenmin I\\
% \Tenmin あ\\
% \Tenmin あ\\
% \Tenmin J\\
% \Tenmin あ\\
% \Tenmin あ\\
% \Tenmin K\\
% \Tenmin あ\\
% \Tenmin あ\\
% \Tenmin L\\
% \Tenmin あ\\
% \Tenmin あ\\
% \Tenmin M\\
% \Tenmin あ\\
% \Tenmin あ\\
% \Tenmin N\\
% \Tenmin あ\\
% \Tenmin あ\\
% \Tenmin O\\
% \Tenmin あ\\
% \Tenmin あ\\
% \Tenmin P\\
% \Tenmin あ\\
% \Tenmin あ\\
% \Tenmin Q\\
% \Tenmin あ\\
% \Tenmin あ\\
% \Tenmin R\\
% \Tenmin あ\\
% \Tenmin あ\\
% \Tenmin S\\
% \Tenmin あ\\
% \Tenmin あ\\
% \Tenmin T\\
% \Tenmin あ\\
% \Tenmin あ\\
% \Tenmin U\\
% \Tenmin あ\\
% \Tenmin あ\\
% \Tenmin V\\
% \Tenmin あ\\
% \Tenmin あ\\
% \Tenmin W\\
% \Tenmin あ\\
% \Tenmin あ\\
% \Tenmin X\\
% \Tenmin あ\\
% \Tenmin あ\\
% \Tenmin Y\\
% \Tenmin あ\\
% \Tenmin あ\\
% \Tenmin Z\\
% \Tenmin あ\\
% \Tenmin あ\\
% \Tenmin a\\
% \Tenmin あ\\
% \Tenmin あ\\
% \Tenmin b\\
% \Tenmin あ\\
% \Tenmin あ\\
% \Tenmin c\\
% \Tenmin あ\\
% \Tenmin あ\\
% \Tenmin d\\
% \Tenmin あ\\
% \Tenmin あ\\
% \Tenmin e\\
% \Tenmin あ\\
% \Tenmin あ\\
% \Tenmin f\\
% \Tenmin あ\\
% \Tenmin あ\\
% \Tenmin g\\
% \Tenmin あ\\
% \Tenmin あ\\
% \Tenmin h\\
% \Tenmin あ\\
% \Tenmin あ\\
% \Tenmin i\\
% \Tenmin あ\\
% \Tenmin あ\\
% \Tenmin j\\
% \Tenmin あ\\
% \Tenmin あ\\
% \Tenmin k\\
% \Tenmin あ\\
% \Tenmin あ\\
% \Tenmin l\\
% \Tenmin あ\\
% \Tenmin あ\\
% \Tenmin m\\
% \Tenmin あ\\
% \Tenmin あ\\
% \Tenmin n\\
% \Tenmin あ\\
% \Tenmin あ\\
% \Tenmin o\\
% \Tenmin あ\\
% \Tenmin あ\\
% \Tenmin p\\
% \Tenmin あ\\
% \Tenmin あ\\
% \Tenmin q\\
% \Tenmin あ\\
% \Tenmin あ\\
% \Tenmin r\\
% \Tenmin あ\\
% \Tenmin あ\\
% \Tenmin s\\
% \Tenmin あ\\
% \Tenmin あ\\
% \Tenmin t\\
% \Tenmin あ\\
% \Tenmin あ\\
% \Tenmin u\\
% \Tenmin あ\\
% \Tenmin あ\\
% \Tenmin v\\
% \Tenmin あ\\
% \Tenmin あ\\
% \Tenmin w\\
% \Tenmin あ\\
% \Tenmin あ\\
% \Tenmin x\\
% \Tenmin あ\\
% \Tenmin あ\\
% \Tenmin y\\
% \Tenmin あ\\
% \Tenmin あ\\
% \Tenmin z\\
% \Tenmin あ\\
% \Tenmin あ\\
% \glue 1.07391 minus 1.07391\\
% \Tenmin ぁ\\
% \glue 1.07391 minus 1.07391\\
% \Tenmin あ\\
% \Tenmin あ\\
% \Tenmin あ\\
% \Tenmin あ\\
% \Tenmin あ\\
% \glue 1.07391 minus 1.07391\\
% \Tenmin ぃ\\
% \glue 1.07391 minus 1.07391\\
% \Tenmin あ\\
% \Tenmin あ\\
% \Tenmin い\\
% \Tenmin あ\\
% \Tenmin あ\\
% \glue 1.07391 minus 1.07391\\
% \Tenmin ぅ\\
% \glue 1.07391 minus 1.07391\\
% \Tenmin あ\\
% \Tenmin あ\\
% \Tenmin う\\
% \Tenmin あ\\
% \Tenmin あ\\
% \glue 1.07391 minus 1.07391\\
% \Tenmin ぇ\\
% \glue 1.07391 minus 1.07391\\
% \Tenmin あ\\
% \Tenmin あ\\
% \Tenmin え\\
% \Tenmin あ\\
% \Tenmin あ\\
% \glue 1.07391 minus 1.07391\\
% \Tenmin ぉ\\
% \glue 1.07391 minus 1.07391\\
% \Tenmin あ\\
% \Tenmin あ\\
% \Tenmin お\\
% \Tenmin あ\\
% \Tenmin あ\\
% \Tenmin か\\
% \Tenmin あ\\
% \Tenmin あ\\
% \Tenmin が\\
% \Tenmin あ\\
% \Tenmin あ\\
% \Tenmin き\\
% \Tenmin あ\\
% \Tenmin あ\\
% \Tenmin ぎ\\
% \Tenmin あ\\
% \Tenmin あ\\
% \Tenmin く\\
% \Tenmin あ\\
% \Tenmin あ\\
% \Tenmin ぐ\\
% \Tenmin あ\\
% \Tenmin あ\\
% \Tenmin け\\
% \Tenmin あ\\
% \Tenmin あ\\
% \Tenmin げ\\
% \Tenmin あ\\
% \Tenmin あ\\
% \Tenmin こ\\
% \Tenmin あ\\
% \Tenmin あ\\
% \Tenmin ご\\
% \Tenmin あ\\
% \Tenmin あ\\
% \Tenmin さ\\
% \Tenmin あ\\
% \Tenmin あ\\
% \Tenmin ざ\\
% \Tenmin あ\\
% \Tenmin あ\\
% \Tenmin し\\
% \Tenmin あ\\
% \Tenmin あ\\
% \Tenmin じ\\
% \Tenmin あ\\
% \Tenmin あ\\
% \Tenmin す\\
% \Tenmin あ\\
% \Tenmin あ\\
% \Tenmin ず\\
% \Tenmin あ\\
% \Tenmin あ\\
% \Tenmin せ\\
% \Tenmin あ\\
% \Tenmin あ\\
% \Tenmin ぜ\\
% \Tenmin あ\\
% \Tenmin あ\\
% \Tenmin そ\\
% \Tenmin あ\\
% \Tenmin あ\\
% \Tenmin ぞ\\
% \Tenmin あ\\
% \Tenmin あ\\
% \Tenmin た\\
% \Tenmin あ\\
% \Tenmin あ\\
% \Tenmin だ\\
% \Tenmin あ\\
% \Tenmin あ\\
% \Tenmin ち\\
% \Tenmin あ\\
% \Tenmin あ\\
% \Tenmin ぢ\\
% \Tenmin あ\\
% \Tenmin あ\\
% \glue 1.07391 minus 1.07391\\
% \Tenmin っ\\
% \glue 1.07391 minus 1.07391\\
% \Tenmin あ\\
% \Tenmin あ\\
% \Tenmin つ\\
% \Tenmin あ\\
% \Tenmin あ\\
% \Tenmin づ\\
% \Tenmin あ\\
% \Tenmin あ\\
% \Tenmin て\\
% \Tenmin あ\\
% \Tenmin あ\\
% \Tenmin で\\
% \Tenmin あ\\
% \Tenmin あ\\
% \Tenmin と\\
% \Tenmin あ\\
% \Tenmin あ\\
% \Tenmin ど\\
% \Tenmin あ\\
% \Tenmin あ\\
% \Tenmin な\\
% \Tenmin あ\\
% \Tenmin あ\\
% \Tenmin に\\
% \Tenmin あ\\
% \Tenmin あ\\
% \Tenmin ぬ\\
% \Tenmin あ\\
% \Tenmin あ\\
% \Tenmin ね\\
% \Tenmin あ\\
% \Tenmin あ\\
% \Tenmin の\\
% \Tenmin あ\\
% \Tenmin あ\\
% \Tenmin は\\
% \Tenmin あ\\
% \Tenmin あ\\
% \Tenmin ば\\
% \Tenmin あ\\
% \Tenmin あ\\
% \Tenmin ぱ\\
% \Tenmin あ\\
% \Tenmin あ\\
% \Tenmin ひ\\
% \Tenmin あ\\
% \Tenmin あ\\
% \Tenmin び\\
% \Tenmin あ\\
% \Tenmin あ\\
% \Tenmin ぴ\\
% \Tenmin あ\\
% \Tenmin あ\\
% \Tenmin ふ\\
% \Tenmin あ\\
% \Tenmin あ\\
% \Tenmin ぶ\\
% \Tenmin あ\\
% \Tenmin あ\\
% \Tenmin ぷ\\
% \Tenmin あ\\
% \Tenmin あ\\
% \Tenmin へ\\
% \Tenmin あ\\
% \Tenmin あ\\
% \Tenmin べ\\
% \Tenmin あ\\
% \Tenmin あ\\
% \Tenmin ぺ\\
% \Tenmin あ\\
% \Tenmin あ\\
% \Tenmin ほ\\
% \Tenmin あ\\
% \Tenmin あ\\
% \Tenmin ぼ\\
% \Tenmin あ\\
% \Tenmin あ\\
% \Tenmin ぽ\\
% \Tenmin あ\\
% \Tenmin あ\\
% \Tenmin ま\\
% \Tenmin あ\\
% \Tenmin あ\\
% \Tenmin み\\
% \Tenmin あ\\
% \Tenmin あ\\
% \Tenmin む\\
% \Tenmin あ\\
% \Tenmin あ\\
% \Tenmin め\\
% \Tenmin あ\\
% \Tenmin あ\\
% \Tenmin も\\
% \Tenmin あ\\
% \Tenmin あ\\
% \glue 1.07391 minus 1.07391\\
% \Tenmin ゃ\\
% \glue 1.07391 minus 1.07391\\
% \Tenmin あ\\
% \Tenmin あ\\
% \Tenmin や\\
% \Tenmin あ\\
% \Tenmin あ\\
% \glue 1.07391 minus 1.07391\\
% \Tenmin ゅ\\
% \glue 1.07391 minus 1.07391\\
% \Tenmin あ\\
% \Tenmin あ\\
% \Tenmin ゆ\\
% \Tenmin あ\\
% \Tenmin あ\\
% \glue 1.07391 minus 1.07391\\
% \Tenmin ょ\\
% \glue 1.07391 minus 1.07391\\
% \Tenmin あ\\
% \Tenmin あ\\
% \Tenmin よ\\
% \Tenmin あ\\
% \Tenmin あ\\
% \Tenmin ら\\
% \Tenmin あ\\
% \Tenmin あ\\
% \Tenmin り\\
% \Tenmin あ\\
% \Tenmin あ\\
% \Tenmin る\\
% \Tenmin あ\\
% \Tenmin あ\\
% \Tenmin れ\\
% \Tenmin あ\\
% \Tenmin あ\\
% \Tenmin ろ\\
% \Tenmin あ\\
% \Tenmin あ\\
% \glue 1.07391 minus 1.07391\\
% \Tenmin ゎ\\
% \glue 1.07391 minus 1.07391\\
% \Tenmin あ\\
% \Tenmin あ\\
% \Tenmin わ\\
% \Tenmin あ\\
% \Tenmin あ\\
% \Tenmin ゐ\\
% \Tenmin あ\\
% \Tenmin あ\\
% \Tenmin ゑ\\
% \Tenmin あ\\
% \Tenmin あ\\
% \Tenmin を\\
% \Tenmin あ\\
% \Tenmin あ\\
% \Tenmin ん\\
% \Tenmin あ\\
% \Tenmin あ\\
% \glue 1.07391 minus 1.07391\\
% \Tenmin ァ\\
% \glue 1.07391 minus 1.07391\\
% \Tenmin あ\\
% \Tenmin あ\\
% \Tenmin ア\\
% \Tenmin あ\\
% \Tenmin あ\\
% \glue 1.07391 minus 1.07391\\
% \Tenmin ィ\\
% \glue 1.07391 minus 1.07391\\
% \Tenmin あ\\
% \Tenmin あ\\
% \Tenmin イ\\
% \Tenmin あ\\
% \Tenmin あ\\
% \glue 1.07391 minus 1.07391\\
% \Tenmin ゥ\\
% \glue 1.07391 minus 1.07391\\
% \Tenmin あ\\
% \Tenmin あ\\
% \Tenmin ウ\\
% \Tenmin あ\\
% \Tenmin あ\\
% \glue 1.07391 minus 1.07391\\
% \Tenmin ェ\\
% \glue 1.07391 minus 1.07391\\
% \Tenmin あ\\
% \Tenmin あ\\
% \Tenmin エ\\
% \Tenmin あ\\
% \Tenmin あ\\
% \glue 1.07391 minus 1.07391\\
% \Tenmin ォ\\
% \glue 1.07391 minus 1.07391\\
% \Tenmin あ\\
% \Tenmin あ\\
% \Tenmin オ\\
% \Tenmin あ\\
% \Tenmin あ\\
% \Tenmin カ\\
% \Tenmin あ\\
% \Tenmin あ\\
% \Tenmin ガ\\
% \Tenmin あ\\
% \Tenmin あ\\
% \Tenmin キ\\
% \Tenmin あ\\
% \Tenmin あ\\
% \Tenmin ギ\\
% \Tenmin あ\\
% \Tenmin あ\\
% \Tenmin ク\\
% \Tenmin あ\\
% \Tenmin あ\\
% \Tenmin グ\\
% \Tenmin あ\\
% \Tenmin あ\\
% \Tenmin ケ\\
% \Tenmin あ\\
% \Tenmin あ\\
% \Tenmin ゲ\\
% \Tenmin あ\\
% \Tenmin あ\\
% \Tenmin コ\\
% \Tenmin あ\\
% \Tenmin あ\\
% \Tenmin ゴ\\
% \Tenmin あ\\
% \Tenmin あ\\
% \Tenmin サ\\
% \Tenmin あ\\
% \Tenmin あ\\
% \Tenmin ザ\\
% \Tenmin あ\\
% \Tenmin あ\\
% \Tenmin シ\\
% \Tenmin あ\\
% \Tenmin あ\\
% \Tenmin ジ\\
% \Tenmin あ\\
% \Tenmin あ\\
% \Tenmin ス\\
% \Tenmin あ\\
% \Tenmin あ\\
% \Tenmin ズ\\
% \Tenmin あ\\
% \Tenmin あ\\
% \Tenmin セ\\
% \Tenmin あ\\
% \Tenmin あ\\
% \Tenmin ゼ\\
% \Tenmin あ\\
% \Tenmin あ\\
% \Tenmin ソ\\
% \Tenmin あ\\
% \Tenmin あ\\
% \Tenmin ゾ\\
% \Tenmin あ\\
% \Tenmin あ\\
% \Tenmin タ\\
% \Tenmin あ\\
% \Tenmin あ\\
% \Tenmin ダ\\
% \Tenmin あ\\
% \Tenmin あ\\
% \Tenmin チ\\
% \Tenmin あ\\
% \Tenmin あ\\
% \Tenmin ヂ\\
% \Tenmin あ\\
% \Tenmin あ\\
% \glue 1.07391 minus 1.07391\\
% \Tenmin ッ\\
% \glue 1.07391 minus 1.07391\\
% \Tenmin あ\\
% \Tenmin あ\\
% \Tenmin ツ\\
% \Tenmin あ\\
% \Tenmin あ\\
% \Tenmin ヅ\\
% \Tenmin あ\\
% \Tenmin あ\\
% \Tenmin テ\\
% \Tenmin あ\\
% \Tenmin あ\\
% \Tenmin デ\\
% \Tenmin あ\\
% \Tenmin あ\\
% \Tenmin ト\\
% \Tenmin あ\\
% \Tenmin あ\\
% \Tenmin ド\\
% \Tenmin あ\\
% \Tenmin あ\\
% \Tenmin ナ\\
% \Tenmin あ\\
% \Tenmin あ\\
% \Tenmin ニ\\
% \Tenmin あ\\
% \Tenmin あ\\
% \Tenmin ヌ\\
% \Tenmin あ\\
% \Tenmin あ\\
% \Tenmin ネ\\
% \Tenmin あ\\
% \Tenmin あ\\
% \Tenmin ノ\\
% \Tenmin あ\\
% \Tenmin あ\\
% \Tenmin ハ\\
% \Tenmin あ\\
% \Tenmin あ\\
% \Tenmin バ\\
% \Tenmin あ\\
% \Tenmin あ\\
% \Tenmin パ\\
% \Tenmin あ\\
% \Tenmin あ\\
% \Tenmin ヒ\\
% \Tenmin あ\\
% \Tenmin あ\\
% \Tenmin ビ\\
% \Tenmin あ\\
% \Tenmin あ\\
% \Tenmin ピ\\
% \Tenmin あ\\
% \Tenmin あ\\
% \Tenmin フ\\
% \Tenmin あ\\
% \Tenmin あ\\
% \Tenmin ブ\\
% \Tenmin あ\\
% \Tenmin あ\\
% \Tenmin プ\\
% \Tenmin あ\\
% \Tenmin あ\\
% \Tenmin ヘ\\
% \Tenmin あ\\
% \Tenmin あ\\
% \Tenmin ベ\\
% \Tenmin あ\\
% \Tenmin あ\\
% \Tenmin ペ\\
% \Tenmin あ\\
% \Tenmin あ\\
% \Tenmin ホ\\
% \Tenmin あ\\
% \Tenmin あ\\
% \Tenmin ボ\\
% \Tenmin あ\\
% \Tenmin あ\\
% \Tenmin ポ\\
% \Tenmin あ\\
% \Tenmin あ\\
% \Tenmin マ\\
% \Tenmin あ\\
% \Tenmin あ\\
% \Tenmin ミ\\
% \Tenmin あ\\
% \Tenmin あ\\
% \Tenmin ム\\
% \Tenmin あ\\
% \Tenmin あ\\
% \Tenmin メ\\
% \Tenmin あ\\
% \Tenmin あ\\
% \Tenmin モ\\
% \Tenmin あ\\
% \Tenmin あ\\
% \glue 1.07391 minus 1.07391\\
% \Tenmin ャ\\
% \glue 1.07391 minus 1.07391\\
% \Tenmin あ\\
% \Tenmin あ\\
% \Tenmin ヤ\\
% \Tenmin あ\\
% \Tenmin あ\\
% \glue 1.07391 minus 1.07391\\
% \Tenmin ュ\\
% \glue 1.07391 minus 1.07391\\
% \Tenmin あ\\
% \Tenmin あ\\
% \Tenmin ユ\\
% \Tenmin あ\\
% \Tenmin あ\\
% \glue 1.07391 minus 1.07391\\
% \Tenmin ョ\\
% \glue 1.07391 minus 1.07391\\
% \Tenmin あ\\
% \Tenmin あ\\
% \Tenmin ヨ\\
% \Tenmin あ\\
% \Tenmin あ\\
% \Tenmin ラ\\
% \Tenmin あ\\
% \Tenmin あ\\
% \Tenmin リ\\
% \Tenmin あ\\
% \Tenmin あ\\
% \Tenmin ル\\
% \Tenmin あ\\
% \Tenmin あ\\
% \Tenmin レ\\
% \Tenmin あ\\
% \Tenmin あ\\
% \Tenmin ロ\\
% \Tenmin あ\\
% \Tenmin あ\\
% \glue 1.07391 minus 1.07391\\
% \Tenmin ヮ\\
% \glue 1.07391 minus 1.07391\\
% \Tenmin あ\\
% \par
% \caption{jfm glueと文字幅}
% \label{fig:jfmglue}
% \end{figure}
%
% 
%
% \subsection{和文のスペーシング}
% \label{sec:imp-jspace}
%
% ASCII版日本語\TeX (p\TeX を含む)において,連続する二つの文字$c_1$と$c_2$
% の間に挿入されるglue $g(c_1,c_2)$は,以下の情報に基づいて定まる。
%
% \begin{itemize}
% \item
% $c_i$がいわゆる半角文字であるか($c_i\in A$),あるいは全角文字であるか
% ($c_i\in K$)。
%
% \item
% $c_i\in K$の場合,jfmファイルによって定まる$c_i$の左側および右側のglue
% (jfm glue) $l(c_i)$および$r(c_i)$。
%
% \item
% $c_i\in A$の場合,$c_i$の\|\xspcode| $\gamma_a(c_i)$。
% \SpecialIndex{\xspcode}
% 
% \item
% $c_i\in K$の場合,$c_i$の\|\inhibitxspcode| $\gamma_k(c_i)$。
% \SpecialIndex{\inhibitxspcode}
%
% \item
% \+\kanjiskip+の値$s_k$および\|\xkanjiskip|の値$s_a$。
% \SpecialIndex{\kanjiskip}\SpecialIndex{\xkanjiskip}
% \end{itemize}
%
% なおこの他に\|\noautospacing|によるスペーシング抑止が影響するが,ほとんど
% 使われることがないと想定され,かつ完全にシミュレートすることが不可能である
% ため,常に\|\autospacing|であると仮定して実装を簡略化している\footnote{%
%
% \cs{Underline}等が呼び出された時点で\cs{autospacing}か否かを判断するのは可
% 能であるが,実際に影響するのは\cs{Underline}等を含むパラグラフあるいはbox
% が終了した時点での状況であり,これを正しく推定するのは不可能である。}
%
% (\ref{sec:usage-note}節参照)。
% \SpecialIndex{\noautospacing}\SpecialIndex{\autospacing}
%
% これらの情報に基づいて,$g(c_1,c_2)$は以下のように定まる\footnote{
%
% 特に全角文字が連続した場合の実際の挙動は,これらの式よりもさらに複雑(怪奇)
% であるが,精密なシミュレートは繁雑かつ無意味であるため,ここに示した簡易か
% つ有意な式を用いている。}。
%
% \def\LAND{{\;\land\;}}\def\LOR{{\;\lor\;}}
% \begin{eqnarray}
% c_1\in K\LAND c_2\in K\Longrightarrow{}\hskip-4em\nonumber\\
% g(c_1,c_2)&=&\cases{s_k&if $r(c_1)=l(c_2)=0$\cr
%		      r(c_1)+l(c_2)&otherwise}\label{eq:jg1}\\[.5ex]
% c_1\in K\LAND c_2\in A\Longrightarrow{}\hskip-4em\nonumber\\
% g(c_1,c_2)&=&\cases{r(c_1)&if $r(c_1)\neq0$\cr
%		      s_a&if $r(c_1)=0\LAND\gamma_k(c_1)\in\{1,3\}\LAND
%					   \gamma_a(c_2)\in\{1,3\}$\cr
%		      0&otherwise}\label{eq:jg2}\\[.5ex]
% c_1\in A\LAND c_2\in K\Longrightarrow{}\hskip-4em\nonumber\\
% g(c_1,c_2)&=&\cases{l(c_2)&if $l(c_2)\neq0$\cr
%		      s_a&if $l(c_2)=0\LAND\gamma_a(c_1)\in\{2,3\}\LAND
%					   \gamma_k(c_2)\in\{2,3\}$\cr
%		      0&otherwise}\label{eq:jg3}
% \end{eqnarray}
%
% ここで問題となるのは,文字ごとにjfmファイルで定められている$l(c_i)$と
% $r(c_i)$の値を取得するプリミティブが用意されていないことである。そこで,こ
% れらのglueが0ではないような文字をJISの$1\sim8$区を対象に調査すると,図
% \ref{fig:jfmglue}に示す結果が得られた。図の表記は文字$c$について,1\,zwを
% 1000としたときの;
%
% \begin{quote}
% \fbox{\hbox to1zw{\vphantom{あ}\hss$c$\hss}}\quad
%	$n(l(c)){-}m(l(c))\,/\,w(c)\,/\,n(r(c))-m(r(c))$
% \end{quote}
% であり,$w(c)$は$c$の幅である。また$p(l(c))$, $p(r(c))$は全ての文字につい
% て0であった。なお16区以降の漢字については,これらのglueは0であると信じられ
% る。
%
% 
%
% この図から,$z=1$\,zw, $q$を「”」の幅とすると,以下の関係式が帰納的に求め
% られる。
%
% \begin{eqnarray}
% n(l(c))+w(c)+n(r(c))=z\nonumber\\
% n(l(c))\neq0\LAND n(r(c))\neq0\Longrightarrow{}\nonumber\\
%	n(l(c))&=&m(l(c))=n(r(c))=m(r(c))\label{eq:jfm1}\\
% n(l(c))\neq0\LAND n(r(c))=0\Longrightarrow{}\nonumber\\
%	m(l(c))&=&(z-q)/2\nonumber\\
%	m(r(c))&=&0\label{eq:jfm2}\\
% n(l(c))=0\LAND n(r(c))\neq0\Longrightarrow{}\nonumber\\
%	m(l(c))&=&0\nonumber\\
%	m(r(c))&=&(z-q)/2\label{eq:jfm3}
% \end{eqnarray}
%
% すなわち,$n(l(c))$と$n(r(c))$を求めることができれば,$m(l(c))$と$m(r(c))$
% を算出することができる。
%
% この$n(l(c))$と$n(r(c))$は,glueの挿入を局所的に禁止する\|\inhibitglue|を
% 用いて求めることができる。すなわち$l(c)$と$r(c)$が0であることがわかってい
% る文字(たとえば「あ」)と$c$を連結した\|\hbox|の幅は,\+\inhibitglue+を文
% 字間に挿入した場合には連結の順序によって$n(l(c))$または$n(r(c))$だけ小さく
% なる。つまり\|\hbox|\Meta{str}の幅を\textit{width}(\meta{str})とすると;
% \SpecialIndex{\inhibitglue}
%
% \begin{eqnarray}
% n(l(c))&=&\textit{width}(\hbox{あ\!\meta{$c$}})-
%	    \textit{width}(\hbox{あ\|\inhibitglue|\!\meta{$c$}})
%	\label{eq:jfmleft}\\
% n(r(c))&=&\textit{width}(\hbox{\meta{$c$}\!あ})-
%	    \textit{width}(\hbox{\meta{$c$}\|\inhibitglue|あ})
%	\label{eq:jfmright}
% \end{eqnarray}
% となる。
%
% したがって$g(c_1,c_2)$を求めるための情報は全て入手可能であり,前述の式
% $(\ref{eq:jg1})\sim(\ref{eq:jg3})$によって算出することができる。
%
% 
%
% \subsection{互換性のための処置}
% \label{sec:imp-compat}
%
% \textsf{jumoline.sty}の冒頭では,\LaTeX2e であるか\LaTeX-2.09であるかのチェッ
% クを行ない,\LaTeX2e であれば\|\NeedsTeXFormat|と\|\ProvidesPackage|を
% 宣言する。この部分はバージョン番号/日付をこのマニュアルに整合させるために,
% \textsf{jumoline.dtx}の冒頭にトリッキーな形で埋め込まれており,以下のコードが
% \textsf{docstrip}によって生成される。
% \SpecialIndex{\NeedsTeXFormat}
% \SpecialIndex{\ProvidesPackage}
%
%\begin{quote}
%\begin{verbatim}
%\def\next{LaTeX2e}
%\ifx\fmtname\next
%\def\next{
%\NeedsTeXFormat{LaTeX2e}[1994/12/01]
%\ProvidesPackage{jlet}}
%\else\def\next[#1]{}\fi
%\next
%[2001/05/31 v1.2 ]
%\end{verbatim}
%\end{quote}
%
% \begin{macro}{\PackageError}
% 続いて\textsf{jumoline}の中で唯一使用している\LaTeX2e 固有のマクロである
% \|\PackageError|が未定義であれば(すなわち\LaTeX-2.09であれば),ほぼ同様
% の機能となるように\|\@latexerr|を用いて定義する。
% \SpecialIndex{\@latexerr}
% \end{macro}
%
% \iffalse
%<*package>
% \fi
%    \begin{macrocode}

\ifx\PackageError\undefined
\def\PackageError#1#2#3{\@latexerr{#1:#2}{#3^^J\@ehc}}
\fi

%%^L

%    \end{macrocode}
%
% 
%
% \subsection{レジスタ宣言}
% \label{sec:imp-decl}
%
% \textsf{jumoline}では,以下のレジスタ等を使用する。
%
% \begin{itemize}
% \item
% \begin{macro}{\UnderlineDepth}
% \+\UnderlineDepth+ (dimen)\\
% ベースラインから下線の上端までの距離。デフォルトである\+\strutbox+の深さを
% 参照することを意味する負値(\+-\maxdimen+)に初期化する。
% \end{macro}
% 
% \item
% \begin{macro}{\MidlineHeight}
% \+\MidlineHeight+ (dimen)\\
% ベースラインから中線の下端までの距離。デフォルトである「あ」の高さの1/2を
% 参照することを意味する負値(\+-\maxdimen+)に初期化する。
% \end{macro}
%
% \item
% \begin{macro}{\OverlineHeight}
% \+\OverlineHeight+ (dimen)\\
% ベースラインから上線の下端までの距離。デフォルトである\+\strutbox+の高さを
% 参照することを意味する負値(\+-\maxdimen+)に初期化する。
% \end{macro}
%
% \item
% \begin{macro}{\UMOlineThickness}
% \+\UMOlineThickness+ (dimen)\\
% 下線等の線幅を定める。デフォルトである0.4\,ptに初期化する。
% \end{macro}
%
% \item
% \begin{macro}{\UMO@height}
% \+\UMO@height+ (dimen)\\
% 実際の描線時に使用する,ベースラインから線の上端までの距離。
% \end{macro}
% 
% \item
% \begin{macro}{\UMO@depth}
% \+\UMO@depth+ (dimen)\\
% 実際の描線時に使用する,ベースラインから線の下端までの距離。
% \end{macro}
% 
% \item
% \begin{macro}{\UMO@dqspace}
% \+\UMO@dqspace+ (dimen)\\
% \ref{sec:imp-jspace}節で述べた,全角文字$c$の両側のglueである$l(c)$と
% $r(c)$の算出のために用いられる$(z-q)/2$。ただし$z$は全角スペース幅,$q$は
% 「”」の幅である。
% \end{macro}
% 
% \item
% \begin{macro}{\UMO@tempdim}
% \+\UMO@tempdim+ (dimen)\\
% スクラッチパッド。
% \end{macro}
%
% \item
% \begin{macro}{\UMO@prejfmglue}
% \+\UMO@prejfmglue+ (skip)\\
% 全角文字$c$の処理において,直前の文字$c'$と$c$の間に挿入されるjfm glueを保
% 持する(\ref{sec:imp-jspace}節参照)。すなわち$c'$も全角文字であれば
% $r(c')+l(c)$である。
% \end{macro}
%
% \item
% \begin{macro}{\UMO@postjfmglue}
% \+\UMO@postjfmglue+ (skip)\\
% \ref{sec:imp-jspace}節で述べた,全角文字$c$の右側のglueである$r(c)$を保持
% する。
% \end{macro}
%
% \item
% \begin{macro}{\UMO@mode}
% \+\UMO@mode+ (count)\\
% \meta{text}の要素は,全角文字,半角文字列,あるいはblocked textを単位とし
% て処理される(\ref{sec:imp-word}, \ref{sec:imp-char}節参照)。\+\UMO@mode+
% はある処理単位の直前の処理単位を種類を,以下の識別番号にしたがって保持する。
%
% \begin{itemize}\item[]
% \|\UMO@afterblock|${}=0$\,\ldots\,blocked text
% \|\UMO@afterword |${}=1$\,\ldots\,半角文字列\\
% \|\UMO@afterchar |${}=2$\,\ldots\,全角文字
% \end{itemize}
% \end{macro}
%
% \item
% \begin{macro}{\UMO@spacefactor}
% \+\UMO@spacefactor+ (count)\\
% 半角文字列の処理が終了した時点での\|\spacefactor|の値を保持する。
% \SpecialIndex{\spacefactor}
% \end{macro}
%
% \item
% \begin{macro}{\UMO@firstxspcode}
% \+\UMO@firstxspcode+ (count)\\
% 半角文字列の先頭文字の\+\xspcode+を保持する。
% \SpecialIndex{\xspcode}
% \end{macro}
%
% \item
% \begin{macro}{\UMO@lastxspcode}
% \+\UMO@firstxspcode+ (count)\\
% 半角文字列の末尾文字の\+\xspcode+を保持する。
% \SpecialIndex{\xspcode}
% \end{macro}
%
% \changes{v1.1-1}{2000/07/11}
%	{\cs{UMO@inhibitxspcode}の解説重複を削除。}
% \item
% \begin{macro}{\UMO@inhibitxspcode}
% \+\UMO@inhibitxspcode+ (count)\\
% 処理が完了した全角文字の\+\inhibitxspcode+を保持する。
% \SpecialIndex{\inhibitxspcode}
% \end{macro}
%
% \item
% \begin{macro}{\UMO@prebreakpenalty}
% \+\UMO@prebreakpenalty+ (count)\\
% 半角文字列の先頭文字の\+\prebreakpenalty+を保持する。
% \SpecialIndex{\prebreakpenalty}
% \end{macro}
%
% \item
% \begin{macro}{\UMO@postbreakpenalty}
% \+\UMO@postbreakpenalty+ (count)\\
% 半角文字列の末尾文字の\+\postbreakpenalty+を保持する。
% \SpecialIndex{\postbreakpenalty}
% \end{macro}
%
% \item
% \begin{macro}{\UMO@kpostbreakpenalty}
% \+\UMO@kpostbreakpenalty+ (count)\\
% 処理が完了した全角文字の\+\postbreakpenalty+を保持する。
% \SpecialIndex{\postbreakpenalty}
% \end{macro}
%
% \item
% \begin{macro}{\ifUMO@nospace}
% \+\ifUMO@nospace+ (if)\\
% \meta{text}の先頭要素,あるいはスペース直後のblocked text(およびその連鎖)
% に引き続く要素であることを示す。このような要素については,その左方への語間
% スペース挿入を行なわない。
% \end{macro}
%
% \item
% \begin{macro}{\ifUMO@firstelem}
% \+\ifUMO@firstelem+ (if)\\
% \meta{text}の先頭要素であることを示す。\+\Underline+等の直前の文字との間の
% スペーシングを\TeX に委ねるため,先頭要素については他の要素とは異なる描線
% 手順をとる(\ref{sec:imp-drawline}節参照)。
% \end{macro}
% \end{itemize}
%
%    \begin{macrocode}
%% Register Declaration

\newdimen\UnderlineDepth \UnderlineDepth-\maxdimen
\newdimen\MidlineHeight \MidlineHeight-\maxdimen
\newdimen\OverlineHeight \OverlineHeight-\maxdimen
\newdimen\UMOlineThickness \UMOlineThickness.4pt

\newdimen\UMO@height \newdimen\UMO@depth
\newdimen\UMO@dqspace \newdimen\UMO@tempdim

\newskip\UMO@prejfmglue \newskip\UMO@postjfmglue

\newcount\UMO@mode
\let\UMO@afterblock\z@
\let\UMO@afterword\@ne
\let\UMO@afterchar\tw@

\newcount\UMO@spacefactor
\newcount\UMO@firstxspcode
\newcount\UMO@lastxspcode
\newcount\UMO@inhibitxspcode
\newcount\UMO@prebreakpenalty
\newcount\UMO@postbreakpenalty
\newcount\UMO@kpostbreakpenalty

\newif\ifUMO@nospace \newif\ifUMO@firstelem

%%^L

%    \end{macrocode}
%
% なお上記の他,\LaTeX で宣言された以下のレジスタ等も使用する。
%
% \begin{itemize}\item[]
% \+\@tempboxa+\quad\+\@tempdima+\quad\+\@tempdimb+\quad
% \+\@tempskipa+\quad\+\@tempcnta+\\\+\if@tempswa+
% \SpecialIndex{\@tempboxa}
% \SpecialIndex{\@tempdima}
% \SpecialIndex{\@tempdimb}
% \SpecialIndex{\@tempskipa}
% \SpecialIndex{\@tempcnta}
% \SpecialIndex{\if@tempswa}
% \end{itemize}
%
% 
% 
%
% \subsection{ユーザインタフェースと初期化}
% \label{sec:imp-uifinit}
%
% \begin{macro}{\Underline}
% \begin{macro}{\Midline}
% \begin{macro}{\Overline}
% \begin{macro}{\UMOline}
% ユーザインタフェースであるマクロ\|\Underline|\Meta{text},
% \+\Midline+\Meta{text}, \+\Overline+\Meta{text},
% \+\UMOline+\Meta{height}\Meta{text}は,共通のマクロ;
%
% \begin{quote}
% \+\UMO@line+\Meta{reg-or-def}\Meta{reg-height}\Meta{def-height}\Meta{text}
% \end{quote}
%
% を呼び出して,各マクロに固有の位置に描線する。描かれる線のベースラインに近
% い方の端のベースラインからの高さ$\delta$は,下式によって定まる。
%
% \begin{equation}
% \delta=\cases{\meta{reg-height}&if $\meta{reg-or-def}\geq 0$\cr
%		  \meta{def-height}&if $\meta{reg-or-def}< 0$}
% \label{eq:delta}
% \end{equation}
% したがって各マクロが以下のような引数で\|\UMO@line|を呼び出すことにより,
% \ref{sec:usage-line}節で述べた描線位置の制御が行なわれる。
%
% \begin{center}\begin{tabular}{l|lll}
% &\meta{reg-or-def}&\meta{reg-height}&\meta{def-height}\\\hline
% \+\Underline+&\+\UnderlineDepth+&\+-\UnderlineDepth+&\+\dp\strutbox+\\
% \+\Midline+&\+\MidlineHeight+&\+\MidlineHeight+&{\it height-of\/}(あ)/2\\
% \+\Overline+&\+\OverlineHeight+&\+\OverlineHeight+&\+\ht\strutbox+\\
% \+\UMOline+&\+-\maxdimen+&---&\meta{height}\\
% \end{tabular}\end{center}
% \end{macro}\end{macro}\end{macro}\end{macro}
%
%    \begin{macrocode}
%% User Interface and Initialization

\def\Underline{\UMO@line\UnderlineDepth{-\UnderlineDepth}{-\dp\strutbox}}
\def\Midline{\setbox\@tempboxa\hbox{あ}%
        \UMO@line\MidlineHeight\MidlineHeight{.5\ht\@tempboxa}}
\def\Overline{\UMO@line\OverlineHeight\OverlineHeight{\ht\strutbox}}
\def\UMOline{\UMO@line{-\maxdimen}\z@}

%    \end{macrocode}
%
% \begin{macro}{\UMO@line}
% \begin{macro}{\\}
% \meta{text}の要素として出現する\|\\|は,\LaTeX の原定義とは異なる処理を行
% なう必要があるので,\+\UMO@line+はまずグループを開始し\|\\|を
% \|\UMOnewline|に\|\let|する。
%
% 次に式(\ref{eq:delta})の$\delta$を算出する。この$\delta$は下線等のベースラ
% インに近い方の端からベースラインまでの距離であるので,下式によって線の上端
% の高さ\|\UMO@height| ($h$)と,下端の深さ\|\UMO@depth| ($d$) を算出する。た
% だし$w$は\|\UMOlineThickness|の値である。
%
% \begin{eqnarray}
% h&=&\cases{\delta+w&if $\delta\geq0$\cr
%	     \delta&if $\delta<0$}\\[.3ex]
% e&=&\cases{-\delta&if $\delta\geq0$\cr
%	     -\delta+w&if $\delta<0$}
% \end{eqnarray}
%
% 続いて式(\ref{eq:jg2}), (\ref{eq:jg3})の$(z-q)/2$を算出して\|\UMO@dqspace|
% に設定する。また\meta{text}の先頭要素を処理するため,\+\ifUMO@nospace+と
% \|\ifUMO@firstelem|を真に初期化し,\+\UMO@mode+を\|\UMO@afterblock|とする。
% さらにvertical modeであれば\|\leavevmode|によってhorizontal modeへスイッチ
% する。
%
% \meta{text}は後述する\|\UMO@wordloop|(\ref{sec:imp-word}節参照)によって,
% 語間スペースによって区切られた語要素単位に処理される。この要素を$w_1$,
% \ldots, $w_n$としたとき,\+\UMO@wordloop+には;
%
% \begin{quote}
% $w_1$\verb*| |$w_2$\verb*| |\ldots\verb*| |$w_n$\verb*| \@nil |
% \end{quote}
%
% が渡される。この最終要素\|\@nil|とそれに続くスペースは,
% \+\noexpand\@nil\space+を\|\edef|により展開し,さらにそれを\|\expandafter|
% によって展開するというトリックにより付加される。
%
% \|\UMO@wordloop|によって\meta{text}の処理が完了すると,\+\UMO@line+はグルー
% プを終了し,引き続くテキストとの間のスペーシング等を適切に行なうための
% \|\UMO@aftergroup|を呼び出す(\ref{sec:imp-end}節参照)。
% \end{macro}\end{macro}
%    \begin{macrocode}
\def\UMO@line#1#2#3#4{\begingroup \let\\\UMOnewline
        \relax\ifdim#1<\z@ \UMO@height#3\relax
        \else \UMO@height#2\relax \fi
        \UMO@depth-\UMO@height
        \ifdim\UMO@height<\z@ \advance\UMO@depth\UMOlineThickness
        \else                 \advance\UMO@height\UMOlineThickness \fi
        \settowidth\UMO@dqspace{“}\advance\UMO@dqspace-1zw
        \UMO@dqspace-\UMO@dqspace \divide\UMO@dqspace\tw@
        \UMO@nospacetrue \UMO@firstelemtrue
        \UMO@mode\UMO@afterblock
        \ifvmode\leavevmode\fi
        \def\@tempa{#4 }\edef\@tempb{\noexpand\@nil\space}%
        \expandafter\expandafter\expandafter\UMO@wordloop
                \expandafter\@tempa\@tempb
        \endgroup \UMO@aftergroup}

%%^L

%    \end{macrocode}
%
% 
%
% \subsection{語要素の処理}
% \label{sec:imp-word}
%
% \begin{macro}{\UMO@wordloop}
% \+\UMO@wordloop+は語要素の先頭文字が`\|{|'であるか否かを\|\UMO@ifbgroup|に
% よりチェックし,`\+{+'であればblocked textを処理する\|\UMO@wordblock|を,
% `\+{+'でなければ半角/全角文字列を処理する\|\UMO@iwordloop|を呼び出す。
% \end{macro}
%
% \begin{macro}{\UMO@ifbgroup}
% \begin{macro}{\UMO@ifnc}
% \SpecialIndex{\if@nextchar}
% \+\UMO@ifbgroup+\meta{then}\meta{else}は,
% \+\if@nextchar\bgroup+\meta{then}\meta{else}とほとんど同じであるが,
% `\+{+'の前のスペースを読み飛ばさないことが異なる。これは特に;
%
% \begin{quote}
% \verb*|\Underline{ |\Meta{blocked-text}\ldots\+}+
% \end{quote}
%
% のように,\meta{text}の先頭のスペースを無視しないための措置である。
% \+\UMO@ifbgroup+の実装は,スペース読み飛ばしがないため\|\if@nextchar|より
% も単純であり,\+\futurelet+の簡単な応用例である。
% \end{macro}\end{macro}
%
%    \begin{macrocode}
%% Processing Word Elements

\def\UMO@wordloop{\UMO@ifbgroup\UMO@wordblock\UMO@iwordloop}
\def\UMO@ifbgroup#1#2{\let\@tempa#1\let\@tempb#2\futurelet\@tempc\UMO@ifnc}
\def\UMO@ifnc{\ifx\@tempc\bgroup \let\next\@tempa \else\let\next\@tempb \fi
        \next}
%    \end{macrocode}
%
% \changes{v1.1-1}{2000/07/11}
%	{\cs{ifUMO@nonspace}の誤記修正。}
% \begin{macro}{\UMO@wordblock}
% \+\UMO@wordblock+\Meta{blocked-text}は,まず\|\UMO@spaceskip|
% (\ref{sec:imp-wordspace}節参照)により語要素の直前の語間スペースに対応す
% る下線等を生成し,続いて\|\UMO@putbox|(\ref{sec:imp-drawline}節参照)によっ
% て\!\meta{blocked-text}に下線等を施す。続いて\!\meta{blocked-text}にスペース
% なしで引き続く要素(もしあれば)を\|\UMO@wordloop|で処理するが,その際に前
% 方にスペースを挿入しないように\|\ifUMO@nospace|を真にし,また\|\UMO@mode|を
% \|\UMO@afterblock|にして後続要素が全角文字である場合の文字間スペース挿入を
% 抑止する。一方\!\meta{blocked-text}の直後がスペースである場合は,そのスペー
% スに対する\|\spacefactor|を1000とするために,\+\UMO@spacefactor+を1000にす
% る。
% \end{macro}
%
%    \begin{macrocode}
\def\UMO@wordblock#1{\UMO@spaceskip
        \UMO@putbox\relax{#1}\UMO@nospacetrue \UMO@mode\UMO@afterblock
        \UMO@spacefactor\@m \UMO@wordloop}
%    \end{macrocode}
%
% \begin{macro}{\UMO@iwordloop}
% \begin{macro}{\UMO@theword}
% \+\UMO@iwordloop+\meta{word}\verb*| |は,\meta{word}が\|\@nil|であれば
% \|\UMO@end|(\ref{sec:imp-end}節参照)によって終了処理を行なう。また直前の
% 語要素がblocked textであれば,語要素は\|\UMO@wordblock|により処理されて取
% り除かれているので,\!\meta{word}は空となる。この場合には,次の語要素の処
% 理の際に語間スペース処理を行なう必要があるため,\+\ifUMO@nospace+を偽とし
% て\|\UMO@wordloop|を呼び出す。
%
% 上記以外の場合,\meta{word}の直前のスペース(もしあれば)に対応する下線等
% を\|\UMO@spaceskip|(\ref{sec:imp-wordspace}節参照)により生成し,続いて
% \|\UMO@charloop|\meta{word}\+\@nil+によって\!\meta{word}を構成する文字列の
% 処理を行なう。その際,\|\UMO@mode|を\|\UMO@afterblock|にして,\meta{word}
% の先頭が全角文字である場合の文字間スペース挿入を抑止する。また\meta{word}
% の先頭が半角文字である場合のために,半角文字列を保持する\|\UMO@theword|を
% 空に初期化するとともに,\+\UMO@firstxspcode+を$-1$にして空であることを示す。
% \end{macro}\end{macro}
%
%    \begin{macrocode}
\def\UMO@iwordloop#1 {\def\@tempa{#1}\ifx\@tempa\@nnil \let\next\UMO@end
        \else
        \ifx\@tempa\empty \UMO@nospacefalse
        \else
                \UMO@spaceskip \UMO@mode\UMO@afterblock
                \def\UMO@theword{}\UMO@firstxspcode\m@ne
                \UMO@charloop#1\@nil \fi
        \let\next\UMO@wordloop \fi
        \next}

%%^L

%    \end{macrocode}
%
% 
%
% \subsection{語間スペース処理}
% \label{sec:imp-wordspace}
%
% \begin{macro}{\UMO@spaceskip}
% \+\UMO@spaceskip+は,\+\ifUMO@nospace+が偽である場合に語間スペースに対応す
% る下線等を生成する。語間スペースのためのglue $\sigma$は,
% \ref{sec:imp-solution}節の表\ref{tab:space}に示すように算出される。したがっ
% てスペース直前の語要素生成時点での\|\spacefactor|である\|\UMO@spacefactor|
% に基づき,$\sigma$を算出して\|\@tempskipa|にセットし,
% \+\UMO@skip+\meta{$\sigma$}(\ref{sec:imp-drawline}節参照)によって下線等を
% 生成すればよい。
% \end{macro}
%
% \begin{macro}{\UMO@setspaceskip}
% なおこの$\sigma$算出の過程において,$\+\spacefactor+<2000$かつ
% $\+\spaceskip+\ (s)\neq0$である場合,$s$の幅$n(s)$と伸縮項$p(s)$, $m(s)$を
% 個別に求める必要がある。この操作は\TeX のprimitiveとして用意されていないた
% め,\+\the\spaceskip+により得られる$s$の文字列表現\!\meta{the-s}を
% \|\UMO@setspaceskip|で処理することによって求める。この\!\meta{the-s}は;
% $$
% \def\PLUS{\hbox{\tt\char`\ plus\char`\ }}
% \def\MINUS{\hbox{\tt\char`\ minus\char`\ }}
% \def\LAND{{\;\land\;}}
% \meta{the-s}=\cases{n(s)&			if $p(s)=0\LAND m(s)=0$\cr
%		      n(s)\PLUS p(s)&		if $p(s)\neq0\LAND m(s)=0$\cr
%		      n(s)\MINUS m(s)&		if $p(s)=0\LAND m(s)\neq0$\cr
%		      n(s)\PLUS p(s)\MINUS m(s)&if $p(s)\neq0\LAND m(s)\neq0$}
% $$
% の4通りの表現を持つ。そこで\|\UMO@space|では;
%
% \begin{quote}
% \+\UMO@setspaceskip+\meta{the-s}\verb*| @ @ @ @ \@nil|
% \end{quote}
%
% と呼び出すことにより,少なくとも5個のスペースで区切られた6個の要素(6番目
% 以降は無意味)を\|\UMO@setspaceskip|に与える。なお\meta{the-s}から\+\@nil+
% の直前のスペースまでの部分は,``\+\the\spaceskip\space @ @ @ @ +''を
% \|\edef|により展開し,さらにそれを\|\expandafter|によって展開するというト
% リックにより付加される。
%
% 一方呼び出される\|\UMO@setspaceskip|の引数構文は;
%
% \begin{itemize}\item[]
% \+\UMO@setspaceskip+\meta{$n$}\verb*| |\meta{plus-or-minus}\verb*| |
%	\meta{p-or-m}\verb*| |\meta{minus}\verb*| |\meta{$m$}\verb*| |
%	\meta{trailer}\+\@nil+
% \end{itemize}
%
% であるので,\meta{plus-or-minus}が``\+plus+'', ``\+minus+'', ``\+@+''のいずれ
% であるか,また\meta{minus}が\hbox{``\|minus|''}と``\+@+''のいずれであるか
% を判定することにより,\meta{the-s}がどの表現であるかを調べて各々に対応する
% 伸縮項を持つglueを\|\@tempskipa|に設定することができる。
% \end{macro}
%
% \begin{macro}{\UMO@plus}
% \begin{macro}{\UMO@minus}
% \changes{v1.1-2}{2000/07/11}
% 	{\texttt{plus}と\texttt{minus}を保持するマクロ名のバグ修正。}
%	
% なお,上記の``\+plus+''と``\+minus+''はいずれも\+\catcode+が(11ではなく)
% 12の文字列である。これらを簡単に得るために``\+1pt plus 2pt minus 3pt+''を
% \|\@tempskipa|に設定した上で\|\the\@tempskipa|を行ない,その結果から
% ``\+plus+''と``\+minus+''を抽出して\|\UMO@plus|と\|\UMO@minus|に\|\def|す
% る操作をあらかじめ行なっておく。
% \end{macro}\end{macro}
%
%    \begin{macrocode}
%% Interword Spacing

\def\UMO@spaceskip{\ifUMO@nospace \UMO@nospacefalse \else
        \ifdim\spaceskip=\z@
            \@tempdima\fontdimen3\font\relax
                \multiply\@tempdima\UMO@spacefactor \divide\@tempdima\@m
            \@tempdimb\fontdimen4\font\relax \multiply\@tempdimb\@m
                \divide\@tempdimb\UMO@spacefactor
            \@tempskipa\fontdimen2\font plus\@tempdima minus\@tempdimb\relax
        \else
            \edef\@tempa{\the\spaceskip\space @ @ @ @ }%
            \expandafter\UMO@setspaceskip\@tempa\@nil
        \fi
        \ifnum\UMO@spacefactor<2000\else
                \ifdim\xspaceskip=\z@ \advance\@tempskipa\fontdimen7\font
                \else \@tempskipa\xspaceskip
        \fi\fi
        \UMO@skip\@tempskipa \fi}
\def\UMO@setspaceskip#1 #2 #3 #4 #5 #6\@nil{\@tempdima\z@ \@tempdimb\z@
        \def\@tempa{#2}\def\@tempb{#3}%
        \ifx\@tempa\UMO@plus \@tempdima#3\def\@tempa{#4}\def\@tempb{#5}\fi
        \ifx\@tempa\UMO@minus \@tempdimb\@tempb\relax\fi
        \multiply\@tempdima\UMO@specefactor \divide\@tempdima\@m
        \multiply\@tempdimb\@m \divide\UMO@spacefactor
        \@tempskipa#1 plus\@tempdima minus\@tempdimb\relax}
\def\@tempa#1 #2 #3 #4 #5\@nil{\def\UMO@plus{#2}\def\UMO@minus{#4}}
\@tempskipa1pt plus 2pt minus 3pt
\expandafter\@tempa\the\@tempskipa\@nil

%%^L

%    \end{macrocode}
%
% 
%
% \subsection{文字列の処理}
% \label{sec:imp-char}
%
% \begin{macro}{\UMO@charloop}
% \+\UMO@charloop+は語要素の先頭文字が`\|{|'であるか否かを\|\UMO@ifbgroup|
% (\ref{sec:imp-word}節参照)によりチェックし,`\+{+'であればblocked textを
% 処理する\|\UMO@charblock|を,`\+{+'でなければ半角/全角文字を処理する
% \|\UMO@icharloop|を呼び出す。
% \end{macro}
%
% \begin{macro}{\UMO@charblock}
% \+\UMO@charblock+\Meta{blocked-text}は,まず\|\UMO@putword|
% (\ref{sec:imp-astring}節参照)により\!\meta{block-text}の直前の半角文字列
% (もしあれば)に下線等を施したものを生成し,続いて\|\UMO@putbox|
% (\ref{sec:imp-drawline}節参照)によって\!\meta{blocked-text}に下線等を施
% す。続いて\!\meta{blocked-text}にスペースなしで引き続く要素(もしあれば)
% を\|\UMO@charloop|で処理するが,その際に\|\UMO@mode|を\|\UMO@afterblock|に
% して後続要素が全角文字である場合の文字間スペース挿入を抑止する。一方
% \!\meta{blocked-text}の直後がスペースである場合は,そのスペースに対する
% \|\spacefactor|を1000とするために,\+\UMO@spacefactor+を1000にする。
% \end{macro}
%
%    \begin{macrocode}
%% Processing Characters

\def\UMO@charloop{\UMO@ifbgroup\UMO@charblock\UMO@icharloop}
\def\UMO@charblock#1{\UMO@putword
        \UMO@putbox\relax{#1}\UMO@mode\UMO@afterblock \UMO@spacefactor\@m
        \UMO@charloop}
%    \end{macrocode}
%
% \changes{v1.1-1}{2000/07/11}
%	{節参照の不統一を修正。}
% \begin{macro}{\UMO@icharloop}
% \begin{macro}{\UMO@theword}
% \+\UMO@icharloop+\meta{char}は,\meta{char}が\|\@nil|, \+\UMOspace+,
% \+\UMOnewline+(\+\\+を含む)のいずれかである場合には,まず\|\UMO@putword|
% (\ref{sec:imp-astring}節参照)により\!\meta{char}の直前の半角文字列(もし
% あれば)に下線等を施したものを生成する。続いて\|\@nil|であれば文字列処理を
% 終了,\|\UMOspace|であれば\|\UMO@space|(\ref{sec:imp-spnl}節参照)を呼び
% 出してスペースを挿入,\|\UMOnewline|であれば\|\UMO@newline|
% (\ref{sec:imp-spnl}節参照)を呼び出して改行,のいずれかを行なう。
%
% 上記以外の場合,\meta{char}の文字コードをチェックし,256未満であれば
% \!\meta{char}を\|\UMO@theword|の末尾に付加する。またその際,
% \|\UMO@firstxspcode|が負であれば,すなわち\!\meta{char}が半角文字列の先頭
% であれば,\+\UMO@firstxspcode+と\|\UMO@prebreakpenalty|に\meta{char}の
% \|\xspcode|と\|\prebreakpenalty|を設定して,半角文字列の左側のスペース/ペ
% ナルティ挿入処理で参照できるようにする。さらに(\+\UMO@firstxspcode+の値に
% 関わらず)\meta{char}が文字列の末尾である場合に備えて,\+\UMO@lastxspcode+
% と\|\UMO@postbreakpenalty|に\meta{char}の\|\xspcode|と\|\postbreakpenalty|
% を設定して,半角文字列の右側のスペース/ペナルティ挿入処理で参照できるよう
% にする。
%
% 一方\meta{char}の文字コードが256以上であれば全角文字とみなし,まず
% \|\UMO@putword|により\!\meta{char}の直前の半角文字列(もしあれば)に下線等
% を施したものを生成する。続いて\|\UMO@putchar|\Meta{char}
% (\ref{sec:imp-kchar}節参照)によって\!\meta{char}に下線等を施したものを生
% 成する。また\meta{char}の直後がスペースである場合に備えて,
% \+\UMO@spacefactor+を1000に設定する。\end{macro}\end{macro}
%
% \begin{macro}{\UMOspace}
% \begin{macro}{\UMOnewline}
% なお\|\UMOspace|と\|\UMOnewline|が,\|\UMO@icharloop|の引数以外のコンテク
% ストで出現すると(たとえば\|\Underline|等の外側での出現),適切な処理を行
% なうことができない。そこでこれらは\|\PackageError|によってエラーメッセージ
% 出力などのエラー処理を行なうように定義される。
% \end{macro}\end{macro}
%
%    \begin{macrocode}
\def\UMO@icharloop#1{\def\@tempa{#1}%
        \ifx\@tempa\@nnil \UMO@putword \let\next\relax
        \else\ifx\UMOspace#1\relax \UMO@putword \let\next\UMO@space
        \else\ifx\UMOnewline#1\relax \UMO@putword \let\next\UMO@newline
        \else
        \ifnum`#1<256\relax \edef\UMO@theword{\UMO@theword#1}%
                \ifnum\UMO@firstxspcode<\z@
                        \UMO@firstxspcode\xspcode`#1\relax
                        \UMO@prebreakpenalty\prebreakpenalty`#1\relax
                        \fi
                \UMO@lastxspcode\xspcode`#1\relax
                \UMO@postbreakpenalty\postbreakpenalty`#1\relax
        \else \UMO@putword \UMO@putchar{#1}\UMO@spacefactor\@m\fi
        \let\next\UMO@charloop \fi\fi\fi \next}
\def\UMOspace{\PackageError{jumoline}%
        {\string\UMOspace\space cannot be used here.}%
        {\string\UMOspace\space can be used only in the argument of
                \string\Underline\space and its relatives.}}
\def\UMOnewline{\PackageError{jumoline}%
        {\string\UMOnewline\space cannot be used here.}%
        {\string\UMOnewline\space can be used only in the argument of
                \string\Underline\space and its relatives.}}

%%^L

%    \end{macrocode}
% 
%
% \subsection{半角文字列の処理}
% \label{sec:imp-astring}
%
% \begin{macro}{\UMO@putword}
% \begin{macro}{\UMO@theword}
% \+\UMO@putword+は\+\UMO@theword+が空でない場合,そこに格納された半角文字列
% $w$に下線等を施すとともに,その左側への文字間スペース/改行ペナルティ
% 挿入を行なう。まず\|\UMO@mode|をチェックし,その値が\|\UMO@afterchar|であ
% る場合,すなわち$w$の直前の文字が全角文字$k$である場合には,
% ペナルティと文字間スペースを挿入する。ペナルティの値は$k$の
% \|\postbreakpenatly|である\|\UMO@kpostpenalty|と,$w$の先頭文字$a$の
% \|\prebreakpenalty|である\|\UMO@prebreakpenalty|の和である。また文字間スペー
% スは式(\ref{eq:jg2})で定まり,$k$の右側のjfm glue ($r(k)$)である
% \|\UMO@postjfmglue|が0でなければその値である。一方0であれば,$\gamma_k(k)$
% である\|\UMO@inhibitxspcode|と,$\gamma_a(a)$である\|\UMO@firstxspcode|が
% ともに1または3であるときに\|\xkanjiskip| ($s_a$)が挿入される。
%
% \begingroup\hfuzz0.6pt
% 次に$w$の直後にスペースがある場合に備えて,$w$からなる\|\hbox|を仮生成して
% \|\spacefactor|を得て,その値を\|\UMO@spacefactor|に設定する。この操作によっ
% て,$w$の末尾文字の\|\sfcode|が0である場合の特例など,\TeX の
% \|\spacefactor|の機構を簡単にシミュレートできる。
% \par\endgroup
%
% 続いて\+\UMO@putbox+(\ref{sec:imp-drawline}節参照)によって$w$に下線を施
% したものを生成する。最後に,\+\UMO@mode+を\|\UMO@afterword|として後続文字
% が全角文字である場合に適切なスペーシングが行なわれるようにし,
% \+\UMO@theword+と\|\UMO@firstxspcode|を空と$-1$に初期化する。
% \end{macro}\end{macro}
%
%    \begin{macrocode}
%% Put ASCII String

\def\UMO@putword{\ifx\UMO@theword\empty\else
        \ifnum\UMO@mode=\UMO@afterchar
                \advance\UMO@kpostbreakpenalty\UMO@prebreakpenalty
                \penalty\UMO@kpostbreakpenalty
                \ifdim\UMO@postjfmglue>\z@ \UMO@skip\UMO@postjfmglue
                \else\ifodd\UMO@inhibitxspcode \ifodd\UMO@firstxspcode
                        \UMO@skip\xkanjiskip \fi\fi\fi\fi
        \setbox\@tempboxa\hbox{%
                \UMO@theword\global\UMO@spacefactor\spacefactor}%
        \UMO@putbox\relax\UMO@theword \UMO@mode\UMO@afterword
        \def\UMO@theword{}\fi \UMO@firstxspcode\m@ne}

%%^L

%    \end{macrocode}
%
% 
%
% \subsection{全角文字の処理}
% \label{sec:imp-kchar}
%
% \begin{macro}{\UMO@putchar}
% \begin{macro}{\UMO@zskip}
% \+\UMO@putchar+\meta{$c$}は,まず$c$の左側への文字間スペース/改行ペナルティ
% を挿入し,次に$c$に下線等を施し,最後に$c$の右側への文字間スペース/改行ペ
% ナルティ挿入の準備をする。
%
% まず$c$の左右への文字間スペース/改行ペナルティ挿入の準備として,
% \|\UMO@mode|が\|\UMO@afterchar|であれば,すなわち$c$の直前の文字$c'$が全角
% 文字であれば,$c'$の右側のjfm glue ($r(c')$)である\|\UMO@postjfmglue|を
% \|\UMO@prejfmglue|に設定する。一方$c'$が全角文字でなければ
% \|\UMO@prejfmglue|を0とする。さらに\|\UMO@postjfmglue|を0に初期化した後,
% $c$がJISの15区までの文字であれば後述の\|\UMO@setjfmglue|\meta{$c$}によって,
% \|\UMO@prejfmglue|に$r(c')+l(c)$を(式(\ref{eq:jg1})参照),
% また$c$の右側の文字間スペース挿入の準備として\|\UMO@postjfmglue|に$r(c)$を,
% それぞれ設定する。続いて\|\@tempskipa| ($g(c',c)$)に\|\UMO@prejfmglue|を,
% \|\UMO@inhibitxspcode|に$c$の\|\inhibitxspcode|を,改行ペナルティ ($p$)を
% 保持する\|\@tempcnta|に$c$の\|\prebreakpenalty|をそれぞれ設定する。
%
% 次に\|\UMO@mode|が\|\UMO@afterchar|であれば,すなわち$c'$が全角文字であれ
% ば,$p$に$c'$の\|\postbreakpenalty|である\|\UMO@kpostbreakpenalty|を加え,
% さらに$r(c')+l(c)$である\|\UMO@prejfmglue|が0であれば$g(c',c)$を
% \|\kanjiskip| ($s_k$)とする(式(\ref{eq:jg1})参照)。
%
% 一方\|\UMO@mode|が\|\UMO@afterword|であれば,すなわち$c'$が半角文字であれ
% ば,$p$に$c'$の\|\postbreakpenalty|である\|\UMO@postbreakpenalty|を加える。
% また式(\ref{eq:jg3})にしたがって,$l(c)$である\|\UMO@prejfmglue|が0でなく,
% かつ$\gamma_a(c')$のである\|\UMO@lastxspcode|と$\gamma_k(c)$である
% \|\UMO@inhibitxspcode|がともに2または3であれば,$g(c',c)$を\|\xkanjiskip|
% ($s_a$)とする。
%
% 続いてペナルティ$p$を挿入した後,$g(c',c)$の全ての項が0でなければ$g(c',c)$
% を挿入する。なおskipに対する\|\ifdim|は幅のみを対象とするので,あらかじめ
% 全項が0のskipの\|\the|表現を\|\UMO@zskip|に\|\def|しておき,$g(c',c)$の
% 全項0の判定は\|\the\@tempskipa|と\|\UMO@zskip|を\|\ifx|で比較することによ
% り行なう。
%
% 続いて\+\UMO@putbox+(\ref{sec:imp-drawline}節参照)によって$c$に下線を施
% したものを生成するが,その際に\TeX によって不要なスペーシングが行なわれる
% のを抑止するため,$c$の両側に\|\inhibitglue|を置くことを指示する。
%
% 最後に$c$の右側の改行ペナルティ挿入の準備として,
% \|\UMO@kpostbreakpenalty|に$c$の\|\postbreakpenalty|を設定し,\|\UMO@mode|
% を\|\UMO@afterchar|とする。
% \end{macro}\end{macro}
%
%    \begin{macrocode}
%% Put Kanji Letter

\def\UMO@putchar#1{%
        \ifnum\UMO@mode=\UMO@afterchar \UMO@prejfmglue\UMO@postjfmglue
        \else                          \UMO@prejfmglue\z@ \fi
        \UMO@postjfmglue\z@
        \ifnum`#1<\kuten"1001\relax\UMO@setjfmglue{#1}\fi
        \@tempskipa\UMO@prejfmglue
        \UMO@inhibitxspcode\inhibitxspcode`#1\relax
        \@tempcnta\prebreakpenalty`#1\relax
        \ifnum\UMO@mode=\UMO@afterchar
                \advance\@tempcnta\UMO@kpostbreakpenalty
                \ifdim\UMO@prejfmglue=\z@ \@tempskipa\kanjiskip \fi
        \else\ifnum\UMO@mode=\UMO@afterword
                \advance\@tempcnta\UMO@postbreakpenalty
                \ifdim\UMO@prejfmglue=\z@
                \ifnum\UMO@lastxspcode>\@ne \ifnum\UMO@inhibitxspcode>\@ne
                        \@tempskipa\xkanjiskip \fi\fi\fi\fi\fi
        \penalty\@tempcnta
        \edef\@tempa{\the\@tempskipa}\ifx\@tempa\UMO@zskip\else
                \UMO@skip\@tempskipa \fi
        \UMO@putbox\inhibitglue{#1}%
        \UMO@kpostbreakpenalty\postbreakpenalty`#1\relax
        \UMO@mode\UMO@afterchar}
\@tempskipa\z@
\edef\UMO@zskip{\the\@tempskipa}
%    \end{macrocode}
%
% \begin{macro}{\UMO@setjfmglue}
% \+\UMO@setjfmglue+\meta{$c$}は,\ref{sec:imp-jspace}節に示した方法に基づき,
% $c$の左右のjfm glueである$l(c)$と$r(c)$を求める。まず式(\ref{eq:jfmleft}), 
% (\ref{eq:jfmright})に従って$n(l(c))$と$n(r(c))$を求め,続いてこれらに基づ
% き式$(\ref{eq:jfm1})\sim(\ref{eq:jfm3})$に従って$m(l(c))$と$m(r(c))$を求め
% る。最終的に$l(c)$は直前の文字$c'$についての$r(c')$を保持している
% \|\UMO@prejfmglue|に加えられ,$r(c)$は\|\UMO@postjfmglue|に格納される。
% \end{macro}
%
%    \begin{macrocode}
\def\UMO@setjfmglue#1{%
        \settowidth\@tempdima{あ#1}\settowidth\@tempdimb{あ\inhibitglue#1}%
        \advance\@tempdima-\@tempdimb
        \settowidth\UMO@tempdim{#1あ}\settowidth\@tempdimb{#1\inhibitglue あ}%
        \advance\UMO@tempdim-\@tempdimb
        \ifdim\@tempdima>\z@
                \ifdim\UMO@tempdim>\z@
                        \@tempskipa\@tempdima minus\@tempdima\relax
                        \UMO@postjfmglue\UMO@tempdim minus\UMO@tempdim\relax
                \else   \@tempskipa\@tempdima minus\UMO@dqspace\relax \fi
                \advance\UMO@prejfmglue\@tempskipa
        \else \UMO@postjfmglue\UMO@tempdim minus\UMO@dqspace \fi}

%%^L

%    \end{macrocode}
%
% 
%
% \subsection{描線処理}
% \label{sec:imp-drawline}
%
% \begin{macro}{\UMO@putbox}
% \+\UMO@putbox+\meta{inh}\Meta{cont}は,\meta{cont}からなる\|\hbox|と,
% 高さが\|UMO@height|, 深さが\|\UMO@depth|, 幅が\|\hbox|\Meta{cont}の幅であ
% るような下線等を生成する。下線等は\|\rlap|または\|\llap|によって幅0の
% \|\hbox|として生成されるが,それを\|\hbox|\Meta{cont}の左右どちらに置くか
% は\|\ifUMO@firstelem|の真偽,すなわち\!\meta{cont}が\!\meta{text}の先頭要
% 素か否かによって定まる。すなわち,先頭要素であれば下線等は\|\llap|によって
% 右側に,そうでなければ\|\rlap|によって左側に置かれる。この結果,
% \+\Underline+等の直前のテキストと最初の\|\hbox|\Meta{cont}の間, および最後
% の\|\hbox|\Meta{cont}と直後のテキストとの間には,下線等のための\|\hbox|が
% 存在しなくなる。p\TeX を含む日本語ASCII-\TeX では;
%
% \begin{quote}
% $c_1\+\hbox+\hbox{\arg{$c_2\ldots$}}\ldots
%	\+\hbox+\hbox{\arg{$\ldots c'_1$}}c'_2$
% \end{quote}
%
% のような系列に対して,$c_1$と$c_2$および$c'_1$と$c'_2$の間のスペーシングを
% 行なうので\footnote{
%
% \texttt{http://www.ascii.co.jp/pb/ptex/}によれば,スペーシングが行なわれな
% いことになっているが,実際は行なわれる。},
%
% \+\Underline+等の両側のスペーシングは(ほぼ)適切となる。
%
% なお\!\meta{inh}は,全角文字については\|\inhibitglue|, それ以外については
% \|\relax|であり,全角文字に対する\TeX でのスペーシングが抑止される
%
% \footnote{
% この結果\!\meta{text}の先頭/末尾要素に関するスペーシングも抑止されるよう
% に見えるが,実際にはそうならならない(理由は不明)。}。
% \end{macro}
%
% \begin{macro}{\UMO@skip}
% \+\UMO@skip+\Meta{glue}は,\ref{sec:imp-solution}節に示した方法に基づき,
% \+\leaders+を用いて高さが\|UMO@height|, 深さが\|\UMO@depth|で,伸縮項を含
% めた幅が\!\meta{glue}であるような下線等を描く。
% \end{macro}
%
%    \begin{macrocode}
%% Draw Under/Mid/Overline

\def\UMO@putbox#1#2{\setbox\@tempboxa\hbox{#1#2#1}\@tempdima\wd\@tempboxa
        \ifUMO@firstelem\else
        \rlap{\vrule\@height\UMO@height\@depth\UMO@depth\@width\@tempdima}\fi
        \box\@tempboxa
        \ifUMO@firstelem \UMO@firstelemfalse
        \llap{\vrule\@height\UMO@height\@depth\UMO@depth\@width\@tempdima}\fi}
\def\UMO@skip#1{%
        \leaders\hrule\@height\UMO@height\@depth\UMO@depth\hskip#1\relax}

%%^L

%    \end{macrocode}
%
% 
%
% \subsection{スペース/改行制御}
% \label{sec:imp-spnl}
%
% \begin{macro}{\UMO@space}
% \begin{macro}{\UMO@sspace}
% \begin{macro}{\UMO@ispace}
% \begin{macro}{\UMO@iispace}
% \begin{macro}{\UMO@penalty}
% ユーザインタフェース\+\UMOspace+\Meta{glue}\opt{pen}に対応する処理は,
% \+\UMOspace+直後の`\+*+'の存在チェック(\+\UMO@space+), 
% \opt{pen}直前の`\+*+'の存在チェック(\+\UMO@ispace+), 
% および\opt{pen}の存在チェック(\+\UMO@iispace+)を含めて行なわれ,
% \meta{glue}に対応する下線等は\|\UMO@skip|(\ref{sec:imp-drawline}節参照)
% によって描かれる。また改行ペナルティは,\+\UMO@penalty+により\LaTeX 仕様
% (\+\@tempswa+が真)または\TeX 仕様(\+\@tempswa+が偽)に従った値で挿入さ
% れる。
%
% 一方\|\UMOspace*|\Meta{glue}に対応する処理は\|\UMO@sspace|が行ない,
% \LaTeX の\|\hspace*|と同様に幅0の\|\vrule|と\|\nobreak|に続いて,
% \|\UMO@skip|によって\!\meta{glue}に対応する下線等を描く。
% 
% なおいずれの場合も\|\UMO@mode|は\|\UMO@aftergroup|となり,処理後に
% \|\UMO@charloop|を呼び出して文字列処理を続行する。
% \end{macro}\end{macro}\end{macro}\end{macro}\end{macro}
%
%    \begin{macrocode}
%% Explicit Spacing and Line Breaking

\def\UMO@space{\UMO@mode\UMO@afterblock
        \@ifstar\UMO@sspace\UMO@ispace}
\def\UMO@sspace#1{\vrule width\z@\nobreak\UMO@skip{#1}\UMO@charloop}
\def\UMO@ispace#1{\@tempskipa#1\relax
        \@ifstar{\@tempswafalse\UMO@iispace}{\@tempswatrue\UMO@iispace}}
\def\UMO@iispace{\@ifnextchar[%]
                              {\UMO@penalty}%
                              {\UMO@skip\@tempskipa \UMO@charloop}}
\def\UMO@penalty[#1]{\@tempcnta#1\relax
        \if@tempswa
                \ifnum\@tempcnta<\z@ \@tempcnta-\@tempcnta \fi
                \ifcase\@tempcnta \or
                        \@tempcnta\@lowpenalty \or
                        \@tempcnta\@medpenalty \or
                        \@tempcnta\@highpenalty \else
                        \@tempcnta\@M \fi
                \ifnum#1<\z@ \@tempcnta-\@tempcnta \fi \fi
        \penalty\@tempcnta \UMO@skip\@tempskipa \UMO@charloop}

%    \end{macrocode}
%
% \begin{macro}{\UMO@newline}
% ユーザインタフェース\|\UMOnewline|(\+*+)の処理は\|\UMO@newline|が行ない,
% `\+*+'がある場合には\|\UMO@skip|により伸長項が\|1fil|の下線等を,また
% `\+*+'がない場合には単に\|\1fil|の空白を挿入し,\|\break|によって改行する。
% なおいずれの場合も\|\UMO@mode|は\|\UMO@aftergroup|となり,処理後に
% \|\UMO@charloop|を呼び出して文字列処理を続行する。
% \end{macro}
%
%    \begin{macrocode}
\def\UMO@newline{\UMO@mode\UMO@afterblock
        \@ifstar{\UMO@skip{0pt plus1fil}\break \UMO@charloop}%
                                 {\hfil \break \UMO@charloop}}

%%^L

%    \end{macrocode}
%
% 
%
% \subsection{終了処理}
% \label{sec:imp-end}
%
% \begin{macro}{\UMO@end}
% \begin{macro}{\UMO@aftergroup}
% \meta{text}の処理が全て完了すると\|\UMO@end|が呼び出され,\meta{text}の末
% 尾文字に関する処理を行なう。すなわち\|\UMO@mode|が\|\UMO@afterword|であれ
% ば\|\spacefactor|に\|\UMO@spacefactor|を設定し,\|\UMO@afterchar|であれば
% \|\UMO@kpostpenalty|と\|\UMO@postjfmglue|の挿入(いずれも0でない場合のみ)
% を行なう。なおこれらの内,\|\spacefactor|の設定は\|\UMO@line|
% (\ref{sec:imp-uifinit}節参照)によるグルーピングの外部で行なう必要がある
% ため,この操作を行なうマクロ\|\UMO@aftergroup|を\|\xdef|し,それを
% \|\UMO@line|がグループ終了後に呼び出す形で実装されている。
% \end{macro}\end{macro}
%
%    \begin{macrocode}
%% Finalization

\def\UMO@end{\ifnum\UMO@mode=\UMO@afterchar
        \ifnum\UMO@kpostbreakpenalty>\z@
                \penalty\UMO@kpostbreakpenalty \fi
        \ifdim\UMO@postjfmglue>\z@
                \UMO@skip\UMO@postjfmglue\fi \fi
        \xdef\UMO@aftergroup{\ifnum\UMO@mode=\UMO@afterword
                \spacefactor\number\UMO@spacefactor\fi}}
%    \end{macrocode}
% \iffalse
%</package>
% \fi
%
% 
%
% \IndexPrologue{\newpage\section*{索引}
% イタリックのページは項目の仕様や用例を,下線のページは項目の実装を,それぞ
% れ示す。また\cs{@}, \cs{UMO@}, \cs{ifUMO@}ではじまるcontrol sequenceについ
% ては,それらを除去してから検索されたい。}
% \Finale
% \GlossaryPrologue{\newpage\section*{改訂履歴}}
% \PrintChanges
\endinput
